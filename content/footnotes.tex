\section{Footnotes}


\trans{
$^{(1.1)}$ Princippet at lagre programmet for løsningen af en opgave i datamatens almindelige indre lager bør efter hvad Maurice Wilkes i 1967 oplyser i sin ACM "Turing Lecture rettelig tilskrives Eckert og Mauchly, fædrene til den første elektroniske regnemaskine. 
}{
$^{(1.1)}$ Anmerkung von Naur zum Text (verlinkt in Kapitel 1.1 und 4.2): Das Prinzip, das Programm zur Lösung einer Aufgabe in einem allgemeinen internen Speicher der Datenmaschine zu speichern, sollte, so Maurice Wilkes 1967 in seiner ACM Turing Lecture, zu Recht Eckert und Mauchly, den Vätern der ersten elektronischen Rechenmaschine, zugeschrieben werden.

$^{(1.2)}$ Sinngemäße Beschreibung. Im Original wird gesagt, dass sich die Firma mit dem Automaten "traf". Als Hintergrund: Remington Rand ist eine Firma, die 1950 die Eckert–Mauchly Computer Corporation gekauft hat und zusammen mit Eckert und Mauchly Maschinen der Art Univac entwickelt und zum Verkauf angeboten hat. Die Firma besaß später eine Abteilung mit dem Namen Univac. 

$^{(1.3)}$ 100 dänischer Kronen im Jahre 1960 entsprechen etwa 1550 dänischer Kronen im Jahre 2024. Der Wert der Maschinen betrüge sich daher heute auf jeweils etwa 15-50 Millionen dänischer Kronen oder ca. 2-7 Millionen Euro.}

$^{(1.4)}$ Geglättete Übersetzung. Naur spricht hier bildlich von einem Strecken in die Umgebung der Lösung. Gemeint ist wohl die Fertigstellung der \enquote{korrekten} (also vom Nutzer eigentlich gewünschten) Berechnung -- im Gegensatz zur Lösung, die durch die Programmierung erreicht wird und die von der Datenmaschine korrekt bzw. folgerichtig mit den Anweisungen ausgerechnet wird.

$^{(1.5)}$ Stark geglättete Übersetzung. Naur versucht hier, das Konzept höherer Programmiersprachen mit Elementen wie Syntaxanalyse und Compilern einzuführen. Eine wörtlichere Übersetzung des Abschnittes lautet: Die  allgemein anwendbare Lösung des Problems der Kommunikation zwischen Mensch und Datenautomat besteht darin, den Datenautomaten auf den Menschen treffen zu lassen, indem man ihn mit der stets komplizierteren Aufgabe der Analyse und Umwandlung menschlicher Äußerungen beauftragt.

$^{(1.6)}$
Geglättete Übersetzung. Naur spricht hier bildlich von "Fahrten mit begrenzter Dauer". 

$^{(1.7)}$ Man beachte, dass der Text 1967 erschien und das Patent auf die erste Computermaus (\enquote{X-Y-Positions-Anzeiger für ein Bildschirmsystem}) erst 1970 für den Erfinder Douglas C. Engelbart genehmigt wurde.

$^{(1.8)}$ Gemeint ist Dänemark, das Heimatland des Autors und Austragungsort der Vorlesung.

$^{(1.9)}$ Im Original \enquote{Akademiet for de tekniske Videnskaber}, abgekürzt ATV.

$^{(1.10)}$ Aus dem Original geht nicht hervor, ob es sich bei diesem Betrag um dänische oder schwedische Kronen handelt. Die Größenordnungen sind aber vergleichbar: 1 Euro entspricht 2025 etwa 7,46 dänischer und 10,90 schwedischer Kronen. 2 Millionen 1947-SEK entsprächen heute etwa 50,5 Millionen 2024-SEK (ca. 4,6 Millionen 2025-Euro) -- 2 Millionen 1947-DDK entsprächen heute etwa 48,9 Millionen 2024-DKK (ca. 6,6 Millionen 2025-Euro).

$^{(1.11)}$ Das mittlerweile nicht mehr existierende Mathematikmaschineninstitut in Stockholm hieß im schwedischen Original \enquote{Matematikmaskinnämnden}.

$^{(1.12)}$ Der mittlerweile nicht mehr existierende Verteidigungsforschungsrat war zur Zeit des kalten Krieges ein Teil der dänischen Streitkräfte und hieß im dänsichen Original \enquote{Forsvarets Forskningsråd}.

$^{(1.13)}$ Im dänischen Original \enquote{militære counterpart-bevilling}, eine Kostenstelle Dänemarks im kalten Krieg.

%https://web.archive.org/web/20141122174912/http://datamuseum.dk/wiki/RC
$^{(1.14)}$ Regnecentralen (Aussprache mit deuscher Lautschrift etwa: Reinezentralen); übersetzt Rechenzentrale oder (sinngemäß) Rechenzentrum. Das Regnecentralen Institut an der Akademie für technische Wissenschaften wurde 1952 gegründet und 1955 mit Geldern aus dem Marshallplan in eine Aktiongesellschaft umgewandelt. Der Name wurde dabei beibehalten. Bekannt wurde die Firma durch den Verkauf von Lochstreifenlesern. Nach wirtschaftlichen Problemen wurde sie 1989 von der britischen firma International Computers Limited übernommen, die 2002 von der japanischen Firma Fujitsu übernommen wurde. Der Name Regnecentralen wurde aber bereits seit 1993 nicht mehr benutzt.
%\todo{Recherchieren: Die deutsche und englische Wikipedia und das Buch geben verschiedene Zeitachsen an! Dänische Originaldokumente sind vorhanden, siehe Link.} 

$^{(1.15)}$ Die dänischen Regionen (im Original \enquote{staten}) sind vergleichbar mit den deutschen Bundesländern und stellen die (einzige) Zwischenstufe zwischen der dänischen Zentralverwaltung und den dänischen Kommunen dar. 

$^{(2.1)}$ Vergleiche die Wort-für-Wort Definition des dänischen Original im Glossar.

$^{(3.1)}$
1 Million 1967-DKK entsprechen ca. 10,5 Millionen 2024-DKK (ca. 1,4 Millionen 2024-Euro)

$^{(3.2)}$
Bei Ferritkernen handelt es sich um eine keramische Mischung aus Eisenoxid und anderen Metall- oder Metalloxiden. Diese Materialien werden zu einem Pulver gemahlen, das dann zu einem festen Kern geformt wird und magnetische Eigenschaften (Ferrit-Magnetismus) aufweist. Kernspeicher basierend auf Ferritkernen wurden etwa von 1954 bis 1975 als Speicher in damaligen Computern eingesetzt.

$^{(4.1)}$ Das dänische Wort im Original ist \enquote{strikkeopskrifter}, also wortwörtlich \enquote{Strickanleitung}, weswegen der Vergleich im Original noch besser funktioniert.

$^{(5.1)}$
Die \enquote{Danmarks tekniske Bibliotek} war eine Einrichtung in Lyngby, einem Vorort von Kopenhagen. Sie wurde 1995 in \enquote{Danmarks Tekniske Videncenter og Bibliotek} (Dänemarks technisches Videocenter und Bibliothek) und 2008 in \enquote{Danmarks Tekniske Informationscenter} (Dänemarks technisches Informationscenter) umbenannt. Im Jahre 2013 wurde die Einrichtung mit der \enquote{Ingeniørhøjskolen i København} (Ingenieurshochschule in Kopenhagen) zur \enquote{Danmarks Tekniske Universitet} (Dänemarks technischer Univesität) fusioniert.

$^{(5.2)}$
Karen Brixen (1885-1962) war eine dänische Schriftstellerin, die in Deutschland meinst unter ihrem Pseudonym Tania Blixen und in England unter Isak Dinesen bekannt ist. Sie leitete 17 Jahre lang eine Kaffeefarm in Kenia. Internationale Bekanntheit erlangte ihr autobiographisches Werk \enquote{Den afrikanske farm} (wörtlich \enquote{Die afrikanische Farm}, verkauft als \enquote{Jenseits von Afrika}), das im Jahre 1985 vom Regisseur Sydney Pollack unter dem Titel \enquote{Out of Africa} verfilmt wurde.

$^{(5.3)}$
Hierbei handelt es sich um den \enquote{Treaty Banning Nuclear Weapon Tests in the Atmosphere, in Outer Space and Under Water, August 5, 1963}, der am 05.08.1963 von den USA, dem UK, und der UdSSR in Moskau unterzeichnet wurde. Ein analoger Text unter dem Namen \enquote{Partial Nuclear Test Ban Treaty} wurde am 08.08.1963 von der DDR, am 09.08.1963 von Dänemark, und am 19.08.1963 von der BRD unterzeichnet. Der Text ist aus dem Artikel 1b des Vertrages und online verfügbar unter \url{https://avalon.law.yale.edu/20th_century/usmu015.asp}. Rechtlich bindend sind für alle Unterzeichner nur die englische und russische Version des Textes.

$^{(5.4)}$
Naur hat im dänischen Original den Satz ins dänische übersetzt. Der Satz lautet auf deutsch, in sprachlich geglättet, etwa: Es wird in diesem Zusammenhang (von allen Beteiligten) verstanden, dass die Bestimmungen in diesem Unterabsatz den Abschluss eines Vertrages nicht ausschließen, der zum Verbot aller Kernwaffentestexplosionen führt, einschließlich unterirdischen Explosionen

$^{(6.1)}$
In den 1960er Jahren gab es zahlreiche Kontroversen bzgl. der Datenhaltung in dänischen Verwaltungen. So kamen die Nationalsozialisten (insb. GeStaPo und SS) zum Beispiel über das Archiv der jüdischen Gemeinde Dänemarks an Namen und Adressen fast aller Juden in Dänemark. Über zentralisierte Daten über politische Straftäter der dänischen Polizei erfuhr man (z.T. bereits vor der Invasion Dänemarks durch Nazideutschland) von der kommunistischen Gesinnung zahlreicher dänischer Staatsbürger. Während viele Juden in einer einmaligen Aktion nach Schweden gerettet werden konnten, wurden bereits sehr früh hunderte hochrangige dänische Kommunisten verhaftet. Weitere historische Informationen sind für geschichtsinteressierte Leser unter anderem im Buch mit der ISBN 3896675109 und in folgendem Journalbeitrag zu finden: \url{https://www.tandfonline.com/doi/full/10.1080/02684527.2021.1976919#d1e341}.

$^{(6.2)}$
600 Millionen 1967-DKK entsprechen ca. 10,4 Millionen 2024-DKK (ca. 1,4 Millionen 2024-Euro).

$^{(6.3)}$
Auch hier ist wird aus dem Original nicht klar, ob es sich um dänische oder norwegische Kronen handelt. Auch hier sind die Größenordnungen jedoch vergleichbar. 50 Millionen 1967-DKK entsprächen etwa 520 Millionen 2024-DKK (ca. 69 Millionen 2024-Euro). In norwegischen Kronen entsprächen 50 Millionen 1967-NOK etwa 612 Millionen 2024-NOK (ca. 52 Millionen 2024-Euro). 