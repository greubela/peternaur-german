\section{Acknowledgement}

Dieses Werk wurde übersetzt von André Greubel im Rahmen des Verbundprojektes ComeMINT. ComeMINT wird im Rahmen der BMBF-Förderlinie \enquote{Kompetenzzentrum für digitales und digital gestütztes Unterrichten in Schule und Weiterbildung im MINT-Bereich} durch das Programm der Europäischen Union \enquote{NextGenerationEU} unter dem Förderkennzeichen \emph{01JA23M06A-N} gefördert. 


\section*{Glossar}

    Naurs Werk wurde zu einer Zeit verfasst, in dem viele der heute üblichen Begriffe für informatische Konzepte noch nicht etabliert waren. Insbesondere an Stellen, an denen Naur versucht, eigene (dänische) Begriffe zu etablieren oder sich von anderen Begriffen abzugrenzen, ist die Übersetzung daher nicht einfach. Insbesondere muss an einiger Stelle entschieden werden, ob ein Begriff auf sein modernes Pendant zurückgeführt werden soll, oder ein heute unüblicher Begriff benutzt wird. In diesem Glossar bieten wir einen Überblick über die wichtigsten Übersetzungsentscheidungen.
    
\subsection{regnemaskiner, analogimaskine}

    Mechanische Maschinen zur Berechnung von Zahlen (wie sie in Kapitel 1.2 beschrieben werden) werden im dänischen Original als \emph{regnemaskiner} (Rechenmaschinen) bezeichnet. Diese wörtliche Übersetzung wurde beibehalten.
    
    Der erste dänische Computer besaß den Eigennamen \emph{analogimaskine} (Analogmaschine). Der Eigenname wurde in Kap. 1.6 wörtlich eingedeutscht.


\subsection{data}
    Der Begriff \emph{data} (Daten) wird im Buch explizit eingeführt: \enquote{Jede beliebige Darstellung von Tatsachen oder Ideen auf eine formalisierte Art und Weise, die durch irgendeinen Prozess kommuniziert oder manipuliert werden kann}. Der Begriff wurde wörtlich übersetzt und wird von Naur auch analog zum modernen Begriff verwendet. 
    
    Die Wort-für-Wort-Übersetzung des dänischen Textes lautet wie folgt: enhver (jede beliebige) repræsentation (Darstellung) af (von) fakta (Tatsachen) eller (oder) ideer (Ideen) på (auf) en (eine) formaliseret (formalisierte) måde (Art), som (die) kan (kann) kommunikeres (kommuniziert werden) eller (oder) manipuleres (manipuliert werden) ved (durch) en eller anden (wörtl. \enquote{einen oder anderen}, dänische Phrase für \enquote{irgendeinen}) proces (Prozess)
    
    Die Übersetzung \enquote{Jede beliebige formalisierte Darstellung...} wäre im deutschen idiomatischer, wurde aber in dieser Übersetzung nicht gewählt, da diese die Satzstruktur verändert und Naur an einigen Stellen einzelne Teile diese Definition analysiert und dies in der Übersetzung ansonsten nicht gleichermaßen hätte dargestellt werden können. Mit dieser Wahl enthält die Definition weder im dänischen noch im deutschen im ersten Teil einen Hinweis auf die Notwendigkeit der formalisierten Darstellung.

\subsection{datamodel, dataprocesser}

    Der Begriff \emph{datamodel} (Datenmodell) wird von Naur benutzt, um ein Modell zu bezeichnen, das aus Daten in einem Datenautomaten. besteht. Naur führt diesen Begriff über Vergleiche mit existierenden Modellen in den Naturwissenschaften ein (2.2) und benutzt ihn konsequent für eine statische (darstellende) Sicht auf Daten. Der Begriff wurde wörtlich als Datenmodell übersetzt.

    Der Begriff \emph{dataprocesser} (Datenprozesse) wird von Naur benutzt, um einen Prozess zu bezeichnen, der Daten verarbeitet. Naur führt diesen Begriff über Vergleiche mit Prozessen in der Industrie (Kap 2.3) und benutzt ihn konsequent für eine dynamische (verarbeitende) Sicht auf Daten. Der Begriff wurde wörtlich als Datenprozesse übersetzt. 

\subsection{Oplysning, Information}

    Der Begriff \emph{oplysning} entstammt aus der Alltagssprache und bedeutet so viel wie Auskunft, Beleuchtung, oder Aufklärung. Naur benutzt den Begriff als Kontrast zu data, um Tatsachen oder Ideen zu bezeichnen, die eben nicht formalisiert dargestellt sind.
    
    Im Text ergibt das Begriffspaar data und oplysning damit einen Kontrast, der dem heutigen deutschen Begriffspaar Daten und Information gleicht. Aus diesem Grund wurde oplysning im Text auch als Information übersetzt, falls es sich auf die nicht formalisierte Darstellung von Tatsachen oder ideen bezieht.
    
    Erwähnenswert ist, dass der Begriff oplysning in der heutigen Fachsprache -- analog zum nähesten deutschen Wort Auskunft -- unpassend wäre und sich auch im heutigen Dänisch das Wort \emph{information} (Information) durchgesetzt hat. 
    
\subsection{Datamaskine, Datamat, datamatisk}

    Naur möchte in seiner Begriffswahl die zuvor existierenden Maschinen mit einem festen Einsatzzweck von den wesentlich flexibleren und gesellschaftlich relevanteren programmierbaren general-purpose Maschinen abgrenzen. Deren wichtigste Eigenschaft ist, dass die Steuerung dieser Maschine durch Daten innerhalb der Maschine selbst geschieht. 

    Naur spricht daher zunächst kurz von der titelgebenden \emph{Datamaskine} (wörtlich Datenmaschine), welches wörtlich übersetzt wurde. Er sagt jedoch schnell, dass er den Bergiff \emph{Datamat} vorzieht. Hierbei handelt es sich um ein Kunstwort aus \emph{data} (Daten) und {automat} (Automat). Dieses Wort wurde im Deutschen nahe des ursprünglichen Kunstwortes als Datenautomat übersetzt.
    
    Die Übersetzung beider Begriffe als Computer wurde abgelehnt, weil es zu weit von der Sprachwahl Naurs gewesen wäre. Eine Verkürzung zu Datmat oder Datenmat wurde abgelehnt, weil es sich im Gegensatz zum dänischen Original eher schwer lesen lässt. 

    Der Begriff \emph{datamatisk} ist die Adjektivform zu datamat. Eine wörtliche Übersetzung wäre datenautomatisch. Dieser Begriff ist im deutschen jedoch sperrig und insb. ist es schwierig, im schriftlichen hier zwischem dem intendierten datenautomat-isch und dem ebenfalls lesbaren daten-automatisch zu unterscheiden.
    
    Aus diesem Grund wurde der Begriff sinngemäß als datenautomatbasiert (z.B. datamatisk proces als datenautomatenbasierter Prozess) bzw. datenautomatenkundig (z.B. datamatisk menneske als datenautomatenkundiger Mensch) übersetzt. 
    
\subsection{Datalogi}

    Naur stellt in diesem Buch fest, dass die automatische Verarbeitung von Daten einerseits hinreichend unterschiedlich von etablierten Wissenschaften ist -- Naur studierte Astronomie und arbeitete später unter anderem am mathematischen Institut an der Universität Cambridge -- und anderseits unabhängig von der Ingenieurskunst bei der Entwicklung konkreter Maschinen gedacht werden muss. Er sprach daher, als einer der ersten, von einem eigenen Wissenschaftsbereich der \enquote{Lehre über Daten und Datenverarbeitung}, welchen er selbst \emph{Datalogi} nannte.

    Tatsächlich wurde Naur 4 Jahre nach Erscheinen des Buches als Professor für Datalogie an die Universität Kopenhagen berufen und gründete somit die erste dänische Forschungseinrichtung in der Informatik.
    
    Eine relevanter Aspekt der Übersetzung war, ob das Wort  aufgrund dieser historischen Bedeutung und der Relevanz im Text wörtlich als Datalogie übersetzt wird -- oder mit dem heutigen Pendant Informatik (welcher sich mittlerweile, gegen den Willen Naurs, auch in Dänemark durchgesetzt hat). Die aktuelle Version des Textes enthält Vorschläge für die historische Variante in \textcolor{blue}{blau} und einem modernen Pendant in \textcolor{orange}{orange}. Die \alt{alte}{neue} Variante ist wie in dieser Zeile zu sehen gesetzt. 
    
    Wichtig ist, dass die Bezeichnung Computerwissenschaft (oder engl. computer science) direkt Naurs Intentionen widersprochen hätte, da sie den Fokus zu sehr auf die die Maschinen selbst legt. 

\subsection{datamatik, databehandling}

    Vereinzelt benutzt Naur, ohne einführende oder weitere Erläuterungen, auch die Begriff \emph{datamatik} (abgeleitet vom Begriff Datamat bzw. datamatisk) und databehandling (Datenbehandlung). Unklar bleibt, warum Naur insgesamt sechs Mal diese -- damals ebenfalls nicht etablierten -- Begriffe benutzt, statt auf andere und häufig benutzte Begriffe zurückzugreifen. Zwei Übersetzungen sind denkbar: wahlweise als mit dem in diesem Kontext gut passenden Begriff der Datenverarbeitung -- oder allgemeiner mit dem Begriff informatisch. In der aktuellen Version sind beide Varianten angegeben.       
    % Überschrift von 1.6, 1x in Kapitel 5 und 2x in Kapitel 6

\subsection{komplikationer}
    Der Begriff \emph{komplikationer} (Komplikation) beschreibt im Text eine komplexe Operation auf Daten, die vorsichtig erstellt werden muss. Der Begriff wurde etwas freier als Funktion übersetzt, ist aber im Original ein gutes Stück näher am Begriff Funktionsweise, als am modernen -- durch Funktionen in Hochsprachen geprägten -- Funktionsbegriff.

\subsection{datalagring}  

    Der Begriff \emph{datalagring} heißt wörtlich übersetzt Datenlagerung. Naurs Nutzung des Begriff im Kontext der Informatik war damals neu und erklärungswürdig, da mit dem Begriff eher Industriehallen als Computersysteme verknüpft waren. Naur war es jedoch primär wichtig (vgl. Kapitel 3), einen Begriff im dänischen zu finden der weniger psychologisierend ist als zum Beispiel der Begriff hukommelse (Erinnerung, im Sinne eines menschlichen Gedächtnisvorgangs). 
    
    Der Begriff wurde im deutschen der besseren Lesbarkeit wegen nicht mit Datenlagerung, sondern mit dem deutschen Pendant Datenspeicherung übersetzt. Der deutsche Begriff speichern bzw. der englische Begriff save im Rahmen von Computersystemen hat in der dänischen Fachsprache kein direktes sprachliches Pendant -- stattdessen ist es auch heute noch üblich, die Begriffe lagre (lagern) und gemme (aufbewahren) weitgehend synonym für diese Vorgänge zu verwenden.

%   \item \textbf{opkald:} Aufruf 

    
\subsection{dannelse, uddannelse, efterdannelse}

    Die Begriffe \emph{dannelse} (Bildung), \emph{uddannelse} (Aus-Bildung), und \emph{efterdannelse} (Nach-Bildung) entsprechen etmylogisch sehr direkt den Begriffen Bildung, Ausbildung und (beruflicher) Weiterbildung und wurden als solche übersetzt.

    Das Begriffstrio wird im Buch jedoch nicht mit der Präzision benutzt, wie ein deutscher Bildungswissenschaftler die Begriffe benutzen würde. Dies liegt einerseits an feinen sprachlichen Unterschieden. Diese wurden im Rahmen der Übersetzung so weit wie möglich behoben. So wurde z.B. in Kapitel 6.8 das dänische uddannelsessystem zum deutschen Bildungssystem (auch wenn es etmylogisch Ausbildungssystem heißt).
    
    Darüber hinaus benutzt Naur die Begriffe als Fachwissenschaftler nicht mit sehr hoher Präzision. So beklagt er zum Beispiel in Kapitel 6.8 den Mangel an Ausbildungsmöglichkeiten. Im Kontext wird jedoch klar, dass es hier nicht primär um Ausbildungsplätze der beruflichen Ausbildung geht -- sondern allgemein schulische und universitäre Bildung, als auch berufliche Aus- und Weiterbildungen etc. gemeint sind.

\subsection{privatlivets fred}

    Die Phrase \emph{privatlivets fred} bedeutet wörtlich übersetzt Privatlebens Freiheit und wurde mit dem heutigen Konzept der Privatsphäre übersetzt. 
    
    
    %\item \textbf{belyses}: Wörtlich \enquote{beleuchtet}, übersetzt als \enquote{veranschaulicht} (S. 36)



    

    %\item \textbf{Datarepræsentationer}: Datendarstellungen
    
    


    %\item \textbf{procesdel}: Wörtlich Prozessteil (des Datenautomaten). Übersetzt als Prozessor. 

\subsection{Anmerkung zur Übersetzung: Fußnote Seite 12/55}
Das Prinzip, das Programm zur Lösung einer Aufgabe im gewöhnlichen internen Speicher des Datenautomaten zu speichern, sollte nach dem, was Maurice Wilkes 1967 in seiner ACM "Turing Lecture" angibt, richtigerweise Eckert und Mauchly, den Vätern der ersten elektronischen Rechenmaschine, zugeschrieben werden.
    

\todo{Rechner für Wechselkurse: Dänemark (ab 1960): https://www.worlddata.info/europe/denmark/inflation-rates.php // https://www.dst.dk/en/Statistik/laer-om-statistik/prisberegner; Schweden (ab 1920): https://www.scb.se/hitta-statistik/sverige-i-siffror/prisomraknaren/;Norwegen https://www.ssb.no/en/kalkulatorer/priskalkulator}