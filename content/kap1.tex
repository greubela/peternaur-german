
\transSec{Datamaskinernes Historie fra Pascal til Nutiden}{Die Geschichte der Datenmaschinen von Pascal bis Heute}

\trans{Det mål jeg har sat mig for denne og de følgende forelæsninger om datamaskinerne og samfundet er at forklare hvorfor datamaskinerne, eller som jeg hellere vil kalde dem, datamaterne, har fået så stor samfundsmæssig betydning, og hvorfor denne betydning må forventes at vokse i endnu ganske uoverskuelig grad.

Denne betydning må betragtes som et kendt faktum, som iøvrigt let kan underbygges med 
konkrete tal, der for eksempel angiver væksten i antallet af datamater og dem der arbejder med dem, og i de kapitaler der er bundet i dem. Oplysninger af denne art vil jeg give hvor de passer i sammenhængen iøvrigt. 

Der er ingen grund til at undre sig over at drøftelsen af disse tings betydning vil kræve adskillige forelæsninger.
Foreteelser der som datamaterne griber dybt ind i samfundet er ikke betinget af en enkelt idé, en enkelt opfindelse, eller et isoleret problem i samfundet.
De fremkommer ved et samspil af mange omstændigheder, både principielle muligheder, bestemte tekniske opfindelser, og en mangfoldighed af samfundsmæssige, deriblandt økonomiske, forhold. I forelæsningerne vil jeg prøve at belyse alle disse forskellige sider af sagen. 

Til en begyndelse vil jeg i dag skitsere noget af den historiske baggrund for den øjeblikkelige situation, i særdeleshed den udvikling af de tekniske opfindelser som fører frem til nutidens elektroniske datamaskiner eller datamater. 
}{
Das Ziel, das ich mir für diese und die folgenden Vorlesungen über Datenmaschinen und die Gesellschaft gesetzt habe, ist zu erklären, warum Datenmaschinen, oder \emph{Datenautomaten} wie ich sie lieber nennen möchte, eine so große gesellschaftliche Bedeutung erlangt haben, und warum man erwarten darf, dass diese Bedeutung noch weiter in einem noch ganz unvorhesehbaren Ausmaß wachsen wird.

Diese Bedeutung muss als bekannte Tatsache betrachtet werden, die sich im Übrigen auch einfach mit konkreten Zahlen belegen lässt, wie zum Beispiel dem Wachstum der Anzahl der Datenautomaten und derer, die mit ihnen arbeiten, sowie dem Kapital das in ihnen gebunden ist. Auskünfte  dieser Art werde ich dort angeben, wo sie in den umgebenden Zusammenhang passen.

Es gibt keinen Grund sich darüber zu wundern, dass die Diskussion über die Bedeutung dieser Dinge mehrere Vorlesungen erfordern wird.
Erscheinungen die, wie die Datenautomaten, tief in die Gesellschaft eingreifen, sind nicht durch eine einzelne Idee, eine einzige Erfindung oder ein isoliertes Problem in der Gesellschaft bedingt.
Sie kommen aus einem Zusammenspiel vieler Umstände hervor: sowohl prinzipiellen Möglichkeiten, bestimmten technischer Erfindungen, als auch einer Vielfalt gesellschaftlicher, darunter auch wirtschaftlicher, Umstände.
In diesen Vorlesungen will ich versuchen, all diese verschiedenen Seiten zu beleuchten.

Heute will ich als Anfang einige historische Hintergründe zur aktuellen Situation skizzieren, insbesondere die Entwicklung von den technischen Erfindungen, die bis zu den heutigen elektronischen Datenmaschinen oder Datenautomaten geführt haben.
}

\transSubSec{Pascal og Leibniz i 1600-tallet}{Pascal und Leibnitz im 17. Jahrhundert}

\trans{
Den tidlige udvikling af datamater kan ikke skelnes fra udviklingen af mekaniske og automatiske apparater til at udføre regninger. Denne udvikling går tilbage til 1642 da Blaise Pascal, den store franske matematiker og filosof, byggede en maskine der kunne lægge sammen og trække fra. Det næste skridt fulgte tredive år senere, i 1671, da den tyske matematiker og filosof Gottfried Wilhelm Leibniz konstruerede en multiplikationsmaskine. 

Disse opfindelser var forløbere for en række andre opfindelser af maskiner, der ved hjælp af tandhjul og andre mekanismer kan udføre regninger. Efterhånden som den almindelige fabrikationsteknik blev forfinet er en række forskellige maskiner af denne art blevet almindeligt kendt fra utallige anvendelser. 
}
{
Die frühe Entwicklung von Datenautomaten kann nicht von der Entwicklung mechanischer und automatischer Apparate zur Durchführung von Rechnungen getrennt werden. Diese Entwicklung geht auf das Jahr 1642 zurück, als Blaise Pascal, der große französische Mathematiker und Philosoph, eine Maschine baute, die addieren und subtrahieren konnte. Der nächste Schritt folgte dreißig Jahre später, im Jahr 1671, als der deutsche Mathematiker und Philosoph Gottfried Wilhelm Leibniz eine Multiplikationsmaschine konstruierte.

Diese Erfindungen waren die Vorläufer einer Reihe anderer Erfindungen von Maschinen, die mithilfe von Zahnrädern und anderen Mechanismen Rechnungen ausführen können. Nach und nach, als die allgemeinen Fertigungstechniken verbessert wurden, wurden eine Reihe verschiedener Maschinen dieser Art allmählich aus unzähligen Anwendungen bekannt.
}


\transSubSec{Charles Babbage — hundrede år forud for sin tid}{Charles Babbage — hundert Jahre seiner Zeit voraus}

\trans{
Fra Leibniz er der et spring på halvandet hundrede år til den næste afgørende udvikling. Den er knyttet til navnet Charles Babbage, og er et ekstremt eksempel på ideer der er forud for deres tid. Charles Babbage var englænder, levede fra 1792 til 1871, og var hele livet optaget af matematiske og tekniske ideer og opfindelser. Omkring 1820 udtænkte han en såkaldt differensmaskine, en regnemaskine der var særlig egnet til at beregne tabeller som de bruges i astronomien og til navigation. Det lykkedes ham at opnå støtte fra den engelske regering og arbejdet med at konstruere maskinen blev sat i gang. Før den var fuldendt fik Babbage dog nye ideer til en langt mere ambitiøs konstruktion, den analytiske maskine. Samtidig viste det sig mere tidsrøvende og kostbart end forudset at konstruere differensmaskinen, og en stor del af de sidste halvtreds år af sit liv tilbragte Babbage med at slås for sine ideer, uden at det lykkedes ham at bringe sine projekter til fuldendelse. I sin egen tid gik Babbage for at være fantast, i dag er det klart at han mere end 100 år før tiden havde indset de muligheder der er realiseret i de moderne datamater. 

Det næste vigtige skridt i udviklingen af mekanisk regning var den praktiske brug af hulkort til at repræsentere tal. Denne ide har rødder helt tilbage til 1725 da den franske væver Bouchon fandt på at bruge en papirrulle med huller til at styre trådene under vævningen. Denne idé blev en af forløberne for den berømte Jacquard væv fra 1808. Til trods for at Babbage allerede omkring 1825 var fuldt klar over hulkortenes muligheder ved regninger, blev denne tanke først virkeliggjort omkring 1880 af Hollerith. Efter dette tidspunkt har alle højt udviklede regnemaskiner gjort brug af huller i papir, hvad enten papiret har haft form af kort eller strimler. 
}
{
Es gibt einen Sprung von etwa huntertfünfzig Jahren von Leibniz bis zur nächsten maßgeblichen Entwicklung. Diese ist mit dem verbunden Namen Charles Babbage und sie ist ein extremes Beispiel für Ideen, die ihrer Zeit voraus sind. Charles Babbage war ein Engländer, lebte von 1792 bis 1871 und beschäftigte sich sein ganzes Leben lang mit mathematischen und technischen Ideen und Erfindungen. Etwa 1820 erfand er eine sogenannte Differenzmaschine, eine Rechenmaschine, die besonders geeignet war zum Berechnen von Tabellen, wie sie in der Astronomie und zur Navigation gebraucht wurden. Es ist ihm geglückt, die Unterstützung der englischen Regierung zu erhalten und die Arbeiten zur Konstruktion der Maschine wurden in Gang gesetzt. Bevor diese vollständig fertig wurde, bekam Babbage jedoch neue Ideen für eine weitaus ehrgeizigere Konstruktion, die analytische Maschine. Gleichzeitig zeigte es sich als zeitaufwändiger und teurer als erwartet, die Differenzmaschine zu konstruieren, und einen großen Teil seiner letzten fünzig Lebensjahre verbrachte Babbage damit, für seine Ideen zu kämpfen, ohne dass es ihm geglückt ist, seine Projekte zur Vollendung zu bringen. In seiner Zeit galt Babbage als Fantasist, heute ist klar dass er mehr als 100 Jahre vor der Zeit die Möglichkeiten moderner Datenautomaten erkannte. 

Der nächste wichtige Schritt in der Entwicklung der mechanischen Rechnung war der praktische Einsatz von Lochkarten zur Darstellung von Zahlen. Die Wurzeln dieser Idee reichen zurück bis 1725, als der französische Weber Bouchon die Verwendung einer Papierrolle mit Löchern erfand, um die Fäden beim Weben zu führen. Diese Idee wurde zu einem der Vorläufer des berühmten Jacquard-Webstuhls von 1808. Obwohl Babbage bereits etwa 1825 volle Klarheit über die Möglichkeiten von Lochkarten beim Rechnen hatte, wurde dieser Gedanke zuerst etwa 1880 durch Hollerith verwirklicht. Ab diesem Zeitpunkt haben alle hoch entwickelten Rechenmaschinen von Löchern im Papier Gebrauch gemacht, egal ob das Papier die Form von Karten oder Streifen hatte.
}

\transSubSec{De elektromekaniske giganters æra}{Die Zeit der elektromechanischen Giganten}

\trans{
Herefter skal vi helt frem til 1930-erne før der igen for alvor sker et spring i udviklingen. Fra omkring 1936 opstår der mellem hænderne på en tysk ingeniør, Konrad Zuse, en række indviklede, automatiske regnemaskiner. De første var rent mekaniske, men allerede fra omkring 1941 overtager elektriciteten de fleste funktioner. Den tredie af disse maskiner, Zuse Z3 fra 1941, kan betragtes som den første funktioneringsdygtige maskine, der realiserer Babbage's ideer, omend med andre midler end forudset i 1822. 

Omtrent samtidig opstår lignende maskiner ved Harvard universitetet og ved Bell Telephone Laboratorierne i U.S.A. Ved Harvard begyndte Howard Aiken i 1937 et samarbejde med firmaet IBM om udvikling af en stor regnemaskine. Maskinen var færdig i 1944 og efterfulgtes de følgende år af yderligere tre store maskiner. Ved Bell Telephone Laboratorierne skabte blandt andre George Stibitz fra 1939 til 1946 seks forskellige komplicerede og ydedygtige maskiner. Alle de nævnte maskiner var opbygget af mekaniske og elektromekaniske dele, tandhjul, vippearme, og elektromekaniske relæer, og deres historiske betydning var at de gjorde det klart at det var muligt og værdifuldt at bygge komplicerede regnemaskiner. 
}{
Hiernach müssen wir bis in die 1930er Jahre gehen, bevor es erneut zu einem ernsthaften Sprung in der Entwicklung kommt. Ab etwa 1936 entstand durch den deutschen Ingenieur Konrad Zuse eine Reihe komplizierter, automatischer Rechenmaschinen. Die erste war rein mechanisch, doch bereits ab etwa 1941 übernahm die Elektrizität die meisten Funktionen. Die dritte dieser Maschinen, die Zuse Z3 von 1941, kann als die erste funktionsfähige Maschine betrachtet werden, die Babbages Ideen verwirklichte -- wenn auch mit anderen Mitteln als 1822 vorgesehen.

Etwa zur gleichen Zeit tauchen ähnliche Geräte an der Harvard University und in den Bell Telephone Laboratories in den USA auf. In Harvard begann Howard Aiken 1937 eine Zusammenarbeit mit der Firma IBM zur Entwicklung einer großen Rechenmaschine. Die Maschine wurde 1944 fertiggestellt und in den folgenden Jahren folgten drei weitere große Maschinen. In den Bell Telephone Laboratories entwickelte unter anderem George Stibitz von 1939 bis 1946 sechs verschiedene komplizierte und leistungsstarke Maschinen. Alle genannten Maschinen bestanden aus mechanischen und elektromechanischen Teilen, Zahnrädern, Kipphebeln und elektromechanischen Relais und ihre historische Bedeutung bestand darin, dass sie deutlich machten, dass es möglich und wertvoll war, komplizierte Rechenmaschinen zu bauen.
}

\transSubSec{Elektroner og programlagring}{Elektronen und Programmspeicherung}

\trans{
Det næste skridt blev taget da Mauchly og Eckert ved University of Pennsylvania i Philadelphia i årene 1944 til 1946 udviklede den første regnemaskine der benytter elektronrør til at realisere de indre funktioner. Herved opnåede man en forøgelse i regnehastighed fra nogle få regneoperationer pr. sekund til flere tusinde. 

Hermed var vejen banet for et skridt der er ejendommeligt derved, at det på en vis måde kun er en gevinst i bekvemmelighed ved brugen af maskinerne, og dog må betegnes som den vigtigste principielle opfindelse på dette område siden Babbage's idé til den analytiske maskine. 
Denne udvikling skyldes matematikerne John Von Neumann og Herman Goldstine fra Princeton universitetet i U.S.A. $^{(1.1)}$ og ideen går ud på at de oplysninger der kræves til at styre maskinens funktioner under løsningen af en opgave, inde i maskinen opbevares på samme måde som de talværdier der bearbejdes. Ideen indebærer at maskinen opbygges omkring et centralt lager for oplysninger, som altså både rummer tal der er ved at blive bearbejdet og instruktioner for maskinens arbejde. 

Dette nye princip for opbygningen af maskiner satte skel mellem alle tidligere maskiner og de fleste senere. Ikke sådan at forstå at de nyere maskiner egentlig kan gøre andet end de ældre, men mere ved den flexibilitet de nye maskiner fik, den lethed hvormed de kan omstilles fra at løse én opgave til en anden. Dette fik konsekvenser langt ud over hvad man umiddelbart kunne vente, og har blandt andet fået os til at indse at disse maskiner slet ikke udelukkende, eller blot overvejende, bør betragtes som regnemaskiner, men at de som vi senere skal omtale nærmere er maskiner til at manipulere data af enhver art. Det er også baggrunden for at vi der arbejder med maskinerne, i dag er tilbøjelige til at føle betegnelsen »elektronisk regnemaskine« som vildledende. Jeg vil derfor i det følgende for de maskiner, der er bygget på princippet at gemme data og procesinstruktioner i det samme indre lager, benytte betegnelsen »datamat«. 

Von Neumann og Goldstine's ideer vandt meget hurtigt forståelse langt ud over verden. Det primære problem blev nu at udvikle teknisk set gode løsninger for det centrale element i datamaterne, lageret for data. I denne henseende blev England i nogle år igen førende. Faktisk var den første egentlige datamat der kunne arbejde tilfredsstillende den såkaldte Edsac, konstrueret af Wilkes og Renwick i Cambridge, og funktioneringsdygtig fra 1949. I Edsac bestod lageret af kviksølvfyldte rør, hvorigennem der cirkulerede lydimpulser svarende til tal og instruktioner. Snart efter fulgte Williams i Manchester med en datamat der byggede på et af ham selv udviklet lagerprincip, det såkaldte Williamsrør. 

Indtil dette tidspunkt var alle datamater enestående konstruktioner, udviklet af en institution, oftest et universitet, til sit eget brug, eller til levering efter kontrakt til en bestemt kunde. Omkring 1950 bliver en stor datamat for første gang tilbudt på det åbne marked. Det var Remington Rand som mødte med maskinen Univac. Firmaet havde til at udvikle denne maskine sikret sig fædrene til den første elektroniske regnemaskine, Eckert og Mauchly. Den første kunde blev det amerikanske folketællingsbureau, men snart efter fulgte et stort forsikringsselskab. Det er vanskeligt at sætte sig ind i den tankegang der dengang dominerede, selv hos kyndige der havde udviklingen nært inde på livet. Så sent som på dette tidspunkt udtalte en kendt ekspert at man næppe kunne vente at der ville blive bygget mere end måske en snes datamater ved universiteter verden over. Hermed, mente man, ville behovet for beregninger være dækket. 

Det skulle gå anderledes. I løbet af få år blev det klart at datamaternes værdi ingenlunde er begrænset til matematiske eller videnskabelige opgaver, men er langt bredere, og i særdeleshed omfatter en fylde af administrative opgaver, i første omgang regnskabsføring af enhver art. Efter nogle års tøven kastede IBM, der hidtil havde koncentreret sig om de simplere hulkortmaskiner og skrivemaskiner, sig ud i konstruktionen af store datamater, hvor de hurtigt blev dominerende på markedet. I løbet af 1950'erne blev konstruktionen af datamater til en storindustri, præget af hård konkurrence. Ikke få firmaer på dette felt er bukket under eller har måttet eliminere deres aktiviteter efter store tab. — Dette har dog ikke kunnet bremse udbredelsen af datamaterne; alene i U.S.A. var der i 1965 mere end 32.000 anlæg, hver til en gennemsnitsværdi af et par millioner kroner. 

}{
Der nächste Schritt erfolgte, als Mauchly und Eckert von der University of Pennsylvania in Philadelphia in den Jahren 1944 bis 1946 die erste Rechenmaschine entwickelten, die Elektronenröhren zur Realisierung der internen Funktionen nutzte. Hierdurch erreichte man eine Zunahme der Rechengeschwindigkeit von wenigen Rechenvorgängen pro Sekunde zu mehreren Tausenden.

Hiermit wurde der Weg für einen Schritt geebnet, der dadurch eigenartig ist, weil er auf eine gewisse Art und Weise nur einen Gewinn in der Bequemlichkeit beim Umgang mit den Maschinen darstellt, und dennoch darf man dies als die wichtigste prinzipielle Erfindung auf diesem Gebiet seit Babbages Idee für die Analysemaschine bezeichnen.
Diese Entwicklung ist den Mathematikern John Von Neumann und Herman Goldstine von der Princeton Universität in den USA zu verdanken $^{(1.1)}$, und die Ideen läuft darauf hinaus, die Anweisungen die zur Steuerung der Maschinenfunktionen beim Lösen einer Aufgabe benötigt werden  werden, in der Maschine selbst aufzubewahren - zusammen mit den Zahlenwerten die verarbeitet werden. Die Idee hat zur Folge, dass die Maschine um einen zentralen Informationsspeicher herum aufgebaut ist, der somit sowohl zu verarbeitende Zahlen als auch Anweisungen für die Arbeit der Maschine enthält.

Dieses neue Prinzip für den Aufbau von Maschinen ist ein zentraler Unterschied zwischen allen früheren Maschinen und den später folgenden. 
Nicht in dem Sinne, dass die neueren Maschinen tatsächlich andere Dinge tun können als die älteren, sondern vielmehr in dem Sinne, wie flexibel die neuen Maschinen waren und wie einfach sie von einer Aufgabe auf eine andere umgestellt werden konnten. Dies führte zu Konsequenzen, die weit über das hinausgingen, was man sofort erwarten konnte, und hat uns unter anderem Einsicht gegeben, dass diese Maschinen gar nicht ausschließlich oder auch überwiegend als Rechenmaschine betrachtet werden sollten. Stattdessen sind diese Maschinen, wie wir später genauer erläutern werden, Maschinen zur Manipulation von Daten jeder Art. Dies ist auch der Hintergrund, warum wir, die mit diesen Maschinen arbeiten, heute geneigt sind, die Bezeichnung \enquote{elektronische Rechenmaschine} als irreführend zu empfinden. Im Folgenden möchte ich daher für die Maschinen,
die auf dem Prinzip aufgebaut sind dass Daten und Prozessanweisungen im gleichen internen Speicher aufbewahrt werden, die Bezeichnung \emph{Datenautomat} benutzen. 

Die Ideen von Neumann und Goldstine erlangten sehr schnell weit über die Welt hinaus Beachtung. Das Hauptproblem war nun, technisch gute Lösungen für das zentrale Element in den Datenautomaten, die Speicherung von Daten, zu entwickeln. In dieser Hinsicht war erneut England für einige Jahre führend. Tatsächlich war der sogenannte Edsac der erste eigentliche Datenautomat, der zufriedenstellend arbeitete. Er wurde von Wilkes und Renwick in Cambridge konstruiert und war ab 1949 funktionstüchtig. Bei Edsac bestand der Speicher aus quecksilbergefüllten Röhren, in denen zirkulierende Schallimpulse zu Zahlen bzw. Instruktionen korrespondieren. Kurz darauf folgte Williams in Manchester mit einem Datenautomaten, der auf Basis eines von ihm selbst entwickelten Speicherprinzips gebaut wurde, die sogenannte Williams-Röhre.

Bis zu diesem Zeitpunkt waren alle Datenmaschinen einzigartige Konstruktionen, die von einer Institution, meistens einer Universität, für den eigenen Gebrauch oder zur Lieferung im Rahmen eines Vertrags an einen bestimmten Kunden entwickelt wurden. Etwa um 1950 wird zum ersten Mal ein großer Datenautomat auf dem freien Markt angeboten. Es war die Firma Remington Rand, die mit den Maschinen Univac auftrat $^{(1.2)}$. Die Firma hatte sich für die Entwicklung dieser Maschinen die Väter für die erste elektronische Rechenmaschine, Eckert und Mauchly, gesichert.
Der erste Kunde wurde das US Census Büro, doch bald folgte eine große Versicherungsgesellschaft. Es ist schwierig, sich in die damals vorherrschende Denkweise zu versetzen, selbst unter Experten, die die Entwicklung zeitnah erlebt haben. Selbst zu diesem Zeitpunkt gab ein namhafter Experte an, man könne kaum erwarten, dass mehr als vielleicht ein Dutzend Datenmaschinen an Universitäten weltweit gebaut würden. Hiermit, meinte man, wäre der Bedarf an Berechnungen gedeckt.

Es sollte anders kommen. Im Lauf von wenigen Jahren wurde klar, dass der Wert von Datenmaschinen keinesfalls auf mathematische oder wissenschaftliche Aufgaben begrenzt war -- sondern viel umfassender ist und insbesondere das Feld administrativer Aufgaben umfasst, insbesondere Buchhaltung jeglicher Art. Nach einigen Jahren des Zögerns warf sich IBM, das sich bisher auf die einfacheren Lochkartenmaschinen und Schreibmaschinen konzentriert hatte, auf die Konstruktion von großen Datenautomaten, wo diese schnell zu Marktführern wurden. Im Lauf der 1950er Jahre wurde die Konstruktion von Datenautomaten zu einer Großindustrie, geprägt von harter Konkurrenz. Nicht wenige Unternehmen aus diesem Bereich sind untergegangen oder mussten ihre Aktivitäten nach hohen Verlusten einstellen. — Dies hat jedoch nicht die Verbreitung von Datenautomaten gebremst; allein in den USA lag deren Anzahl bei mehr als 32.000 Anlagen mit einem durchschnittlichen Wert von ein paar Millionen dänischer Kronen $^{(1.3)}$.
}



\transSubSec{Programmeringens æra}{Die Ära der Programmierung}

% p. 15
\trans{
Der skulle ikke gå mange år efter at datamaterne var begyndt at blive leveret til almindelig brug før der viste sig nye, uventede problemer. Det var muligt for ingeniørerne at udvikle og bygge datamater med stadig forøget arbejdshastighed og stadig forbedret pålidelighed, men samspillet mellem datamaterne og de mennesker der formulerede opgaverne for dem gav brat voksende problemer. Maskinerne var der, men det kneb for brugerne at følge med i at fodre dem med opgaver der var tilstrækkelig fuldstændigt og korrekt formuleret til at maskinerne kunne få lejlighed til at strække ud i gennemførelsen af løsningen. En stadig voksende del af omkostningerne ved at drive datamaterne kom til at skyldes forgæves forsøg med ukorrekte programmer. Fra slutningen af 1950'erne kom udviklingen omkring datamaterne således til at stå i programmeringens tegn. 

Programmeringens problemer vil blive nøjere drøftet i den fjerde forelæsning. Her skal blot antydes at løsningen af problemet ligger nøje på linie med den voksende forståelse af datamaterne som generelle symbolmanipulationsapparater. — Den helt almindeligt anvendelige løsning af problemer med kommunikationen mellem mennesker og datamater er at lade datamaten komme mennesket i møde ved at den pålægges et stadig mere kompliceret arbejde med at analysere og omforme menneskets ytringer. Denne almindelige filosofi fandt først anvendelse på det mest påtrængende problem, det at formulere løsningsmetoder for datamaten selv. I stigende grad gik man over til at tillade dem der skulle formulere 
problemløsningerne for datamaterne at udtrykke sig på en for mennesker bekvem måde. Til dette formål udviklede man særlige skrivemåder, såkaldte programmeringssprog, som for eksempel Algol, Fortran, og Cobol. 

Efter at de mest påtrængende problemer med programmeringen af datamaterne er blevet løst på denne måde har interessen vendt sig mod andre sider af kommunikationen mellem datamaterne og mennesker. Indtil omkring 1960 foregik praktisk talt al kontakt ved hjælp af skrivemaskinelignende apparater. For at meddele sig til datamaten måtte brugeren derfor skrive sine meddelelser på et tastatur, og datamatens svar fremkom igen i form af maskinskrift. Ydermere var den enkelte brugers kontakt med datamaten begrænset til kortvarige kørsler med timers mellemrum. Meget af arbejdet med at udvikle datamaterne er i dag rettet mod at fjerne disse begrænsninger. Det er blevet mere almindeligt at lade datamaterne styre apparater der kan tegne, eller der fremviser billeder som på en fjernsynsskærm. "Tilsvarende er det i dag muligt at lade datamater opfatte menneskers ytringer gennem at aflæse en pegepind der føres af mennesket, ligesom der har været lagt en del arbejde i apparater som kan opfatte menneskers tale. Dertil kommer at de højst udviklede datamatiske systemer af i dag tillader brugerne at være i kontakt med systemet så tit og så længe brugeren behøver det. 
}{
Nachdem angefangen wurde Datenautomaten für den allgemeinen Gebrauch zu liefern, sollte es nicht viele Jahre dauern bevor sich neue, unerwartete Probleme zeigten.
Es war den Ingenieuren möglich, Datenautomaten mit stets höherer Arbeitsgeschwindigkeit und stets besserer Zuverlässigkeit zu entwickeln und zu bauen -- doch das Zusammenspiel zwischen den Datenautomaten und den Menschen, die die Aufgaben für sie formulierten, bereitete wachsende Probleme. Die Maschinen waren da, aber die Benutzer hatten Mühe, sie mit Aufgaben zu füttern, die ausreichend vollständig und korrekt formuliert waren, damit die Maschine die Gelegenheit bekamen, der gewünschten Lösung nahe zu kommen $^{(1.4)}$.
Ein ständig wachender Teil der Kosten einen Datenautomat zu betreiben entstand aufgrund erfolgloser Versuche mit inkorrekten Programmen. Ab dem Ende der 1950er-Jahre standen die Entwicklung rund um die Datenmaschinen somit im Zeichen der Programmierung.

Die Probleme der Programmierung werden in der vierten Vorlesung ausführlicher behandelt. An dieser Stelle genügt es bloß anzudeuten, dass die Lösung des Problems genau auf der Linie mit dem wachsenden Verständnis von Datenautomaten als allgemeine Symbolmanipulationsapparate liegt. — Die  allgemein anwendbare Lösung des Problems der Kommunikation zwischen dem Mensch und dem Datenautomat besteht darin, den Datenautomaten zunächst mit einer (immer komplexer werdenden) Analyse menschlicher Anweisungen zu beauftragen, bevor diese Anweisungen ausgeführt werden. $^{(1.5)}$ Diese gewöhnliche Philosophie fand zuerst Anwendung bei dem dringendstem Problem, der Formulierung von Lösungsmethoden für die Datenautomaten selbst. Im steigenden Umfang ging man dazu über, demjenigen, der die Problemlösungen für den Datenautomaten formulieren sollte, zu erlauben, sich auf eine für den Menschen bequeme Art und Weise auszudrücken. Zu diesem Zweck entwickelte man besondere Schreibweisen, so genannte Programmiersprachen, wie zum Beispiel Algol, Fortran und Cobol.

Nachdem die dringendsten Probleme der Programmierung von Datenautomaten auf diese Weise gelöst wurden, hat sich das Interesse auf andere Aspekte der Kommunikation zwischen Datenautomaten und Menschen verlagert. Bis etwa 1960 fanden praktisch gesprochen alle Kontakte mit Hilfe von schreibmaschinenartigen Geräte statt. Um sich dem Datenautomat mitzuteilen, musste der Benutzer daher seine Mitteilung mit einer Tastatur schreiben, und die Antwort des Datenautomaten erschien wieder in Form von Maschinenschrift.
Darüber hinaus war der Kontakt einiger Benutzers mit dem Datenautomaten begrenzt auf Programmausführungen von kurzer Dauer zwisch denen jeweils Stunden lagen $^{(1.6)}$. Ein Großteil der Arbeit bei der Auseinandersetzung mit Datenautomaten ist heute darauf gerichtet, diese Begrenzung zu entfernen. Es ist inzwischen üblich, dass Datenautomaten Apperaturen steuern, die zeichnen oder Bilder wie auf einem Fernsehbildschirm anzeigen können. Ebenso ist es heute möglich, dass Datenautomaten menschliche Äußerungen begreifen, indem sie einen von einem Menschen gehaltenen Zeigestock ablesen$^{(1.7)}$, gleichsam wurde ein Teil der Arbeit in Appertaruren gesteckt, die menschliche Sprache begreifen können.
Dazu kommt, dass die fortschrittlichsten datenautomatischen Systeme heute den Nutzern erlauben, so oft und so lange im Kontakt mit dem System zu sein, wie sie benötigen.
}

%p. 17
\transSubSec{Datamatikken i Danmark}{Datenverarbeitung in Dänemark}

\trans{
Som supplement til denne datamaternes verdens historie vil jeg i dag slutte med en kort gennemgang af udviklingen her i landet. Den første spæde begyndelse til vor overgang til det datamatiske samfund blev gjort i slutningen af 1946 da Akademiet for de tekniske Videnskaber dannede ATV's regnemaskineudvalg. Den drivende kraft og formand i dette udvalg var professor i matematik ved Danmarks tekniske Højskole, dr. phil. Richard Petersen. I de første år koncentrerede man sig omkring bygningen af en såkaldt analogimaskine. I den samme periode skete der en glimrende udvikling i Sverige, efter at den svenske rigsdag i 1947 havde bevilget 2 millioner kroner til formålet. Denne udvikling blev også til stor nytte for os her i landet, idet en række personlige kontakter førte til at den svenske rigsdag i 1952 tiltrådte at alle planer og erfaringer fra udviklingen af den svenske maskine Besk ville blive stillet frit til rådighed for et dansk udviklingsarbejde under ATV”s regnemaskineudvalg. Den første konsekvens heraf blev at civilingeniør Bent Scharøe Petersen tilbragte et års tid ved Matematikmaskinnämnden i Stockholm. På dette tidspunkt var det danske Forsvarets Forskningsråd også blevet interesseret i sagen, og der var dannet et fællesudvalg med ATV's regnemaskineudvalg. 

Det lå dog tungt med pengene. Danske erhvervsvirksomheder var blottede for interesse, og den danske statslige videnskab blegnede ved talen om de beløb der krævedes. Først i 1955 lykkedes det at få en bevilling på 900.000 kr. af den såkaldte militære counterpart-bevilling, og i slutningen af året kunne man omdanne ATV's regnemaskineudvalg til et selvejende ATV-institut, Regnecentralen. Herefter kom der rask gang i at bygge den danske version af Besk, som blev kendt under navnet Dask. 

Snart efter at Dask var sat i drift fik Regnecentralen en opfordring fra Geodætisk Institut til at indgå et samarbejde om udviklingen af en ny, mindre datamat. Dette samarbejde førte til konstruktionen af Gier, den første helt dansk udviklede datamat. I teknisk henseende betød den et spring frem fra Dask derved at den ikke benytter vakuumrør, men transistorer, i de elektroniske kredsløb. Den første Gier blev leveret i 1960 og derefter har produktionen løbet jævnt, og datamater af denne type er købt af virksomheder både i Danmark og i en række andre europæiske lande. 

Også på området programmeringssprog kom vi ved denne tid godt med herhjemme. Allerede mens Dask var under udvikling var det klart at der her lå store problemer, men også store muligheder. Da der i 1958 kom gang i det internationale samarbejde omkring at fastlægge et fælles programmeringssprog sluttede vi os straks til arbejdet, og jeg selv var medlem af en international kommite der i 1960 fastlagde programmeringssproget Algol 60. 

Sideløbende hermed begyndte der at komme gang i anvendelsen af datamater til rent administrative opgaver og indkøb af udenlandske datamater tog fart. I 1959 dannede staten og en række kommuner interessentskabet Datacentralen med henblik på løsningen af større offentlige registeropgaver, ligesom en række firmaer har lagt dele af deres administration over på 
datamater, enten gennem leje af tid hos et af servicebureauerne eller gennem brug af egen datamat. 

Den tekniske aktivitet er også fortsat. Regnecentralen, der siden 1964 har været et aktieselskab, har således foruden flere mindre apparater udviklet en ny datamat, RC 4000, som udnytter den nyeste mikroelektronik. 

Resultatet af denne udvikling er at antallet af datamater her i landet i 1967 er omkring 180, med hastigt stigende tendens. I forhold til vort indbyggerantal er dette ikke et overvældende antal. Fra et snævert økonomisk synspunkt er tallet derimod ikke urimeligt, og endda måske lidt højt. Problemet er ikke blot at have datamatkapaciteten, men også at have tilstrækkelig kapacitet af kyndige brugere. Dette sidste punkt er i dag den alvorligste flaskehals for udnyttelsen af de muligheder som datamaterne frembyder. Det er også baggrunden for de mangfoldige bestræbelser der i disse år gøres for at få kundskaben om datamaterne bragt ind i alle niveauer af uddannelsen. 

På dette punkt står vi svagt, i første række på grund af den traditionalisme og usmidighed der præger vore højere uddannelsesinstitutioner. Det er klart at indpasningen af forholdsvis kostbart apparatur som datamater i universiteterne og de andre højere læreanstalter vil kræve en administrativ nydannelse. I hvor ringe grad vore højere læreanstalter har været i stand til aktivt og selvstændigt at løse dette problem fremgår blandt andet deraf at det største datamatiske anlæg der for tiden står til disse institutioners rådighed skyldes en tidsbegrænset gave fra et amerikansk datamatfirma. 

En anden grund til vore uddannelsesmæssige svagheder må søges i at det endnu kun i ganske utilstrækkelig grad er erkendt at en indsigt i de principper der ligger bag anvendelsen af datamater bidrager til en helt ny forståelse af et bredt felt af andre fag, hvad enten disse har brug for datamaternes hjælp eller ej. Nøglen til denne forståelse ligger i datalogien, læren om data og dataprocesser, som derfor skal være emnet for den anden forelæsning
}{
Als Ergänzung zu dieser Weltgeschichte der Datenautomaten will ich heute mit einem kurzen Rückblick auf die Entwicklung hier im Lande schließen $^{(1.8)}$. Die ersten Anfänge unseres Übergangs zu einer datenautomatischen Gesellschaft wurden Ende 1946 gemacht, als die Akademie der Technischen Wissenschaften $^{(1.9)}$ den ATV-Rechenmaschinenausschuss gründete. Die treibende Kraft und der Vorsitzende dieses Ausschusses war der Professor für Mathematik an der Technischen Universität von Dänemark, Dr. phil. Richard Petersen. In den ersten Jahren konzentrierte man sich auf den Bau einer so genannten Analogmaschine. Im der gleichen Periode geschah in Schweden eine glänzende Entwicklung, nachdem 2 Millionen Kronen im Jahre 1947 vom schwedischen Reichstag zu diesem Zweck zur Verfügung gestellt wurden $^{(1.10)}$. Diese Entwicklung brauchte auch hierzulande Vorteile, denn eine Reihe von persönlichen Kontakten führte dazu, dass der schwedische Reichstag 1952 beschloss, alle Pläne und Erfahrungen aus der Entwicklung der schwedischen Maschine \enquote{Besk} für die dänische Entwicklungsarbeit im Rahmen des ATV-Rechenmaschinenausschusses zur Verfügung zu stellen. Die erste Konsequenz hiervon war, dass Bauingenier Bent Scharøe Petersen ein Jahr Zeit am Mathematikmaschineninstitut $^{(1.11)}$ in Stockholm verbrachte. Zu diesem Zeitpunkt hatte sich auch der dänische Verteidigungsforschungsrat $^{(1.12)}$  Interesse am Thema gezeigt, und es wurde ein gemeinsamer Ausschuss mit dem ATV-Rechenmaschinenausschuss gebildet. 

Dennoch gab es Probleme mit der Finanzierung. Die dänischen Unternehmen zeigten kein Interesse, und die dänische staatliche Wissenschaft erbleichte bei den Beträgen die benötigt wurden. Erst 1955 glückte eine Bewilligung von 900.000 dänischen Kronen aus der so genannten militärischen Counterpart-Bewilligung $^{(1.13)}$, und Ende des Jahres wurde der ATV-Rechenmaschinenausschuss in ein unabhängiges ATV-Institut Regnecentralen $^{(1.14)}$, umgewandelt. Hiernach wurde die dänische Version von Besk, die unter dem Namen \enquote{Dask} bekannt wurde, schnell gebaut. 

Kurz nach der Inbetriebnahme von Dask erhielt Regnecentralen eine Anfrage des Geodätischen Instituts, bei der Entwicklung eines neuen, kleineren Datenautomaten zusammenzuarbeiten. Diese Zusammenarbeit führte zum Bau von \enquote{Gier}, dem ersten vollständig in Dänemark entwickelten Datenautomaten. In technischer Hinsicht bedeutet dieser ein Sprung vorwärts von Dask, da er keine Vakuumröhren benutzt, sondern Transistoren mit elektrischem Kreislauf. Der erste Gier wurde 1960 ausgeliefert, und seither wurde die Produktion kontinuierlich fortgesetzt, und Datenautomaten dieses Typs wurden von Unternehmen sowohl in Dänemark als auch in einer Reihe anderer europäischer Länder gekauft. 

Auch im Bereich der Programmiersprachen kamen wir zu dieser Zeit hierzulande gut voran. Schon während der Entwicklung von Dask war klar, dass es große Probleme, aber auch große Chancen gab. Als 1958 die internationale Zusammenarbeit begann, um eine gemeinsame Programmiersprache zu etablieren, beteiligten wir uns sofort an der Arbeit, und ich selbst war Mitglied eines internationalen Komitees, das 1960 die Programmiersprache Algol 60 festlegte.

Gleichzeitig hiermit begann die Anwendung von Datenautomaten für rein administrative Aufgaben in Gang zu kommen, und der Einkauf von ausländischen Datenautomaten gewann an Fahrt. 1959 bildeten die Regionen $^{(1.15)}$ und eine Reihe von Kommunen die Partnerschaft \enquote{Datacentralen} mit Hinblick auf die Lösung von großen Registrierungsaufgaben, gleichwohl verlagerte eine Reihe Firmen ihre Verwaltung auf Datenautomaten, indem sie entweder Zeit in einem der Dienstleistungszentren anmieteten oder ihren eigenen Datenautomaten benutzten. 

Die technische Tätigkeit wurde auch fortgesetzt. Regnecentralen, die seit 1964 eine Aktiengesellschaft waren $^{(1.16)}$, hat neben mehreren kleinen Apperaten außerdem einen neuen Datenautomat, den RC 4000, entwickelt, der die neueste Mikroelektronik nutzt.

Das Ergebnis dieser Entwicklung ist, dass die Zahl der Datenautomaten hier im Lande im Jahr 1967 bei etwa 180 lag, mit stark steigender Tendenz. Im Verhältnis zu unserer Einwohnerzahl ist dies keine überwältigende Anzahl. Von einem engen wirtschaftlichen Standpunkt ist die Zahl jedoch nicht unvernünftig, vielleicht sogar ein wenig hoch. Das Problem ist nicht nur die Datenautomatkapazitäten, sondern auch eine ausreichende Anzahl sachkundiger Benutzer. Dieser letzte Punkt ist heute der größte Flaschenhals bei der Nutzung der Möglichkeiten, die Datenmaschinen bieten. Dies ist auch der Hintergrund für die zahlreichen Bestrebungen, die in diesem Jahr unternommen werden, um das Wissen über Datenautomaten in allen Niveaus der Ausbildung einzuführen. 

Auf diesen Punkt stehen wir schwach dar, in erster Linie wegen den des Traditionalismus und der Unflexibilität, die unsere Bildungsinstitutionen prägen. Es ist klar, dass die Integration von verhältnismäßig teuren Apperaturen wie Datenautomaten in die Universitäten und anderen Lehrinstitutionen eine neue Verwaltungsstruktur erfordern wird. Inwieweit unsere höheren Lehrinstitutionen im Stand waren, dieses Problem aktiv und unabhängig zu lösen, ergibt sich unter anderem daraus, dass das größten datenautomatischen Anlagen, die diesen Einrichtungen derzeit zur Verfügung steht, ein zeitlich begrenztes Geschenk einer amerikanischen Computerfirma ist. 

Ein anderer Grund für unsere bildungsmäßige Schwachheit liegt darin, dass immer noch zu wenig erkannt wird, dass die Einsicht in die Prinzipien die hinter der Anwendungen von Datenautomaten liegen zu einem neuen Verständnis von einer Vielzahl anderer Fächer beiträgt, unabhängig davon, ob diese die Hilfe von Datenautomaten benötigen oder nicht. Der Schlüssel zu diesem Verständnis liegt in der \alt{Datalogie}{Informatik}, der Lehre von den Daten und den Datenprozessen, die deshalb das Thema der zweiten Vorlesung sein wird.
}
