
\transSec{DATALOGI, LÆREN OM DATA}{\alt{Datalogie}{Informatik}, Lehren über Daten}

\trans{
I den første af disse forelæsninger stillede jeg Dem i udsigt at besvare spørgsmålet: Hvorfor er de elektroniske datamaskiner, eller datamaterne, som jeg foretrækker at kalde dem, af så væsentlig samfundsmæssig betydning? Jeg gennemgik kort regnemaskinernes historie og skitserede den vældige udvikling de har undergået i de sidste årtier, den udvikling der blandt andet har medført at vi i dag betragter disse apparater som datamanipulatorer, og ikke i første række som regnemaskiner. 
}{
In der ersten dieser Vorlesungen stellte ich Ihnen in Aussicht, die Frage zu beantworten: Warum sind elektronische Datenmaschinen, oder Datenautomaten, wie ich vorziehe sie zu nennen, von so großer gesellschaftlicher Bedeutung? Ich habe einen kurzen Rückblick auf die Geschichte der Rechenmaschinen gegeben und die enorme Entwicklung skizziert, die diese in den letzten Jahrzehnten durchlaufen haben, diese Entwicklung, die unter anderem mit sich brachte, dass wir diese Geräte heute als Datenmanipulatoren und nicht in erster Linie als Rechenmaschinen betrachten. 
}

\transSubSec{Data, datarepræsentationer og dataprocesser}{Daten, Datendarstellungen und Datenprozesse}

\trans{
I dag skal vi beskæftige os med nogle helt andre grunde til datamaternes enestående betydning. Jeg vil skitsere det der ligger i begreberne data og de dertil nøje knyttede datarepræsentationer og dataprocesser, og forsøge at vise at disse begreber i sig rummer et helt nyt syn på en mangfoldighed af menneskets ytringsformer, både matematikken og mere dagligdags foreteelser som sproget. Gennem deres intime forbindelse med disse almenmenneskelige ting bliver datamaternes samfundsmæssige betydning langt mere forståelig. 

For nu med det samme at få det værste overstået, her er den internationale definition af ordet »data«: 

DATA: enhver repræsentation af fakta eller ideer på en formaliseret måde, som kan kommunikeres eller manipuleres ved en eller anden proces. 

Denne definition lyder formentlig ret så tam, eller endda tom, men just deri ligger det uhyre omfattende i begrebet. For at give Dem en mere håndgribelig forståelse vil jeg nu gennemgå nogle eksempler og sammenholde dem med definitionen. 

Lad os begynde med sædvanlig sprogbrug. Som eksempel har jeg her på papiret foran mig skrevet en sætning. Sætningen læser: 

»På væggen hænger en sollysende høstscene« Disse tegn på papiret foran mig er i følge definitionen data. Lad os for at forvisse os herom først overveje den første del af definitionen, der siger: »Enhver repræsentation af fakta eller ideer«. Her er der tale om forholdet mellem sætningen på papiret og visse andre omstændigheder i verden, maleriet, væggen, det at maleriet hænger der, og tanken om maleriets motiv. Det er dette forhold mellem sætningen og virkeligheden, der udtrykkes ved at disse data repræsenterer fakta og ideer. Den anden del af definitionen siger at repræsentationen må være formaliseret på en sådan måde at den kan kommunikeres eller manipuleres ved en eller anden proces. Hvad angår den nævnte proces da kan jeg uden videre nævne en sådan, nemlig den der består i at jeg lader mit blik følge teksten og læser den op for Dem som jeg netop gjorde før. Derigennem har jeg kommunikeret disse data til Dem. "Tilbage står kravet om at repræsentationen skal være formaliseret. Her tænkes der på den omstændighed at en tekst som den omtalte består af enkelte tegn, bogstaverne, der holder sig inden for et meget begrænset sæt af former, nemlig de henved tredive bogstaver i alfabetet, og ydermere at små uregelmæssigheder i det enkelte tegns fremtræden på papiret er uden betydning. Denne omstændighed er betingelsen for at man kan tale om en perfekt kommunikation af teksten. Hvis for eksempel en af mine tilhørere har nedskrevet sætningen så kan vi gå ud fra at blot han er fortrolig med dansk retskrivning så vil hans version af sætningen være aldeles den samme som min, bogstav for bogstav. 

Definitionen af data gemmer dog på endnu mere, gennem det den ikke siger. Den udtaler sig aldeles ikke om nogen bestemt måde data kan eksistere og lader derved alle muligheder stå åbne. Allerede den anførte illustration giver en række eksempler herpå. De primære data findes som tryksværte på papir her foran mig. Alene derigennem at jeg har læst sætningen højt og at den er blevet overført til lytterne gennem radioen er disse samme fakta og ideer blevet repræsenteret på en række andre måder, først som en serie lyde jeg udtaler, derefter som elektriske signaler af flere forskellige former, før den omsider igen omsættes til lyd i Deres højttaler. Hver af disse former kan opfattes som data der repræsenterer de samme fakta og ideer. 

Endnu en væsentlig ting skal bemærkes ved den givne definition af data, nemlig den udtrykgelige tale om processer. En ting eller begivenhed er ikke i sig selv data, men bliver det først når den indgår i en proces hvori dens repræsentation af fakta og ideer er det afgørende. Hvis jeg brænder et beskrevet stykke papir så underkastes skrifttegnere en forbrændingsproces, men dette gør dem ikke i sig selv til data. Dette er derimod tilfældet hvis en person læser dem. Vi ser heraf at databegrebet beskæftiger sig med en måde som mennesker forholder sig på til visse fænomener.
}{
Heute werden wir uns mit einigen ganz anderen Gründen für die einzigartige Bedeutung von Datenautomaten beschäftigen. Ich werde skizzieren, was sich hinter den Begriffen \emph{Daten} und den damit eng verbundenen Begriffen \emph{Datendarstellungen} und \emph{Datenprozessen} verbirgt. Daneben versuche ich zu zeigen, dass diese Begriffe eine an sich völlig neue Sichtweise auf eine Vielzahl menschlicher Ausdrucksformen, sowohl auf die Mathematik, als auch auf alltägliche Phänomene wie die Sprache, enthalten. Durch ihre enge Verbindung mit diesen allgemeinen Dingen wird die gesellschaftliche Bedeutung von Datenautomaten viel mehr verständlich. 

Um jetzt sofort das schlimmste zu überstehen, hier ist die internationale Definition des Wortes \enquote{Daten}: 

DATEN: Jede beliebige Darstellung von Tatsachen oder Ideen auf eine formalisierte Art und Weise, die durch irgendeinen Prozess kommuniziert oder manipuliert werden kann $^{(2.1)}$.

Diese Definition klingt vermutlich recht zahm, oder sogar leer, aber genau darin das äußerst umfassende des Begriffs.
Um Ihnen ein greifbareres Verständnis zu geben, will ich nun einige Beispiele durchgehen und diese mit der Definition vergleichen.

Beginnen wir mit dem gewöhnlichen Sprachgebrauch. Als Beispiel habe ich einen Satz auf das Papier vor mir geschrieben. Der Satz lautet: 

\enquote{An der Wand hängt eine sonnenbeschienene Ernteszene} Diese Zeichen auf dem Papier vor mir entsprechen der Definition von Daten. Lasst uns, um uns dessen zu vergewissern, zuerst über den ersten Teil der Definition nachdenken, welche besagt: \enquote{Jede beliebige Darstellung von Tatsachen oder Ideen}. Hier ist die Sprache von der Beziehung zwischen den Sätzen auf Papier und gewissen anderen Umständen auf der Welt, dem Gemälde, der Wand an der das Gemälde hängt, und dem Gedanken vom Motiv des Gemäldes. Es ist diese Beziehung zwischen dem Satz und der Wirklichkeit, die dadurch ausgedrückt wird, dass diese Daten Tatsachen und Ideen darstellen. Der zweite Teil der Definition besagt, dass die Darstellung auf so eine Weise formalisiert sein muss, dass sie durch irgendeinen Prozess kommuniziert oder manipuliert werden kann.
Was diesen erwähnten Prozess angeht, kann ich ohne weiteres so einen benennen, nämlich den der darin besteht, dass ich meinen Blick den Text folgen lasse und diesen Ihnen vorlese, wie ich es gerade getan habe. Dadurch habe ich Ihnen diese Daten kommuniziert.
Es bleibt die Forderung dass die Darstellung formalisiert sein muss. 
Dies bezieht sich auf den Umstand, dass ein Text wie der erwähnte aus einzelnen Zeichen, den Buchstaben, besteht, die sich in einem sehr begrenzten Satz von Formen, nämlich den fast dreißig Buchstaben des Alphabets, aufhält, und darüber hinaus darauf, dass kleine Unregelmäßigkeiten im Auftreten von einzelnen Zeichen auf dem Papier ohne Bedeutung sind.
Dieser Umstand ist eine Voraussetzung für eine perfekte Kommunikation des Textes. Wenn zum Beispiel einer meiner Zuhörer den Satz aufgeschrieben hat, können wir davon ausgehen, dass, sofern er mit der dänischen Rechtschreibung vertraut ist, seine Version des Satzes Buchstabe für Buchstabe genau mit der meinen übereinstimmt. Die Definition von Daten versteckt jedoch noch mehr durch das, was sie \emph{nicht} sagt. Sie sagt absolut nichts über eine bestimmte Art und Weise in der Daten existieren können, und lässt damit alle Möglichkeiten offenstehen. Bereits die angegebene Illustration gibt eine Reihe von Beispielen hierfür. Die primären Daten finden sich als Druckerschwärze auf dem Papier hier vor mir. Allein dadurch dass ich den Satz laut vorgelesen hätte und dass er zum Hören durch das Radio übertragen worden wäre, wären diese Tatsachen und Ideen auf eine Reihe anderer Art und Weise dargestellt worden, zuerst als eine Serie von Lauten die ich ausspreche, anschließend als elektronische Signale von mehreren unterschiedlichen Formen, bevor diese zu guter Letzt wieder zu Lauten in Ihrem Lautsprechern übersetzt werden. Jede dieser Formen kann als Daten aufgefasst werden, die dieselben Tatsachen und Ideen darstellen. 

Eine weitere wesentliche Bemerkung soll bezüglich dieser gegebenen Definition von Daten gemacht werden, nämlich die ausdrückliche Sprache von \emph{Prozessen}. Ein Ding oder eine Begebenheit ist nicht in sich selbst ein Datum, sondern wird das zuerst, sobald es in einen Prozess hineingeht, in dem seine Darstellung von Tatsachen und Ideen das entscheidende ist. Falls ich ein beschriebenes Stück Papier verbrenne, so unterziehe ich den Schriftzeichen einen Verbrennungsprozess, aber dies macht diese nicht in sich selbst zu Daten. Dies ist hingegen der Fall, wenn eine Person diese liest. Wir sehen hieraus dass sich der Datenbegriff mit einer Art und Weise beschäftigt, wie sich ein Mensch bezüglich diese Erscheiungen verhält.  
}

\transSubSec{Kunst og modeller som data}{Kunst und Modelle als Daten}

\trans{
Databegrebet ligger nær op ad symbol- og modelbegreberne. Ordet symbol bruges både om matematiske symboler, hvor det knap kan skelnes fra data, og i kunsten. Brugen af symboler i kunsten er dog en kompliceret sag. Nok er der tale om at én ting på en ejendommelig måde peger på, eller henviser til, en anden, men der gives ikke nødvendigvis et krav om en formalisering som ved databegrebet. Således er der forskel mellem et maleri og data. Maleriet er ikke formaliseret og ingen kopi af det vil regnes for lige så god som originalen. 

At formelle processer ikke er kunsten uvedkommende fremgår på den anden side af forbindelsen mellem litteraturen og bogtrykkunsten, som må betragtes som en tidlig triumf for mekanisk databehandling. Ved at fremdrage dette eksempel håber jeg også at berolige dem der føler sig uhyggeligt til mode ved de moderne datamater. Vi har alle for længst vænnet os til at omgås særdeles personligt med data som er frembragt gennem mekaniske processer. Der er ingen der føler sig stødt af at kærlighedsdigte er reproduceret ved hjælp af ganske følelsesløse trykkemaskiner. Af og til kan man dog iagttage at der måske er en grænse for hvor vidt vi er parat til at akceptere dette upersonlige mellemled. Jeg selv føler mig snydt når jeg modtager en julehilsen der er trykt og hvor afsenderen ikke engang har umaget sig til at tilføje sin egenhændige underskrift. Fornemmelsen kan dog også vendes om. Nutildags regnes et festtelegram jo for en fuldt ud passende form for hyldest, næsten bedre end et brev, til trods for at telegramteksten som bekendt har været udsat for en langt mere omfattende håndtering af uvedkommende personer og maskiner. 

Databegrebets forbindelse til modelbegrebet er måske ikke ved første øjekast indlysende, da man ved en model normalt tænker på en ting som i sin fremtoning har stærk lighed med det den er en model af, som for eksempel ved modelfly. Modelbegrebet er imidlertid for længst overtaget i de empiriske videnskaber, fysikken, astronomien, biologien, og mange andre, og bruges der om langt mere abstrakte konstruktioner. I astronomien taler man for eksempel om stjernemodeller, der består af kurver eller tabeller, altså data. 
}{
Der Datenbegriff liegt nahe am Symbol- oder Modellbegriff. Das Wort Symbol wird sowohl für mathematische Symbole verwendet, wo es kaum von Daten zu unterscheiden ist, als auch in der Kunst. Der Gebrauch von Symbolen in der Kunst ist jedoch eine komplizierte Sache. Wohl ist die Sprache davon, dass ein Ding sich auf eigentümliche Art und Weise auf ein anderes bezieht oder verweist, aber es gibt keine Notwendigkeit für die Anforderung nach einer Formalisierung wie beim Datenbegriff.
Es gibt also so einen Unterschied zwischen einem Gemälde und Daten. Das Gemälde ist nicht formalisiert, und keine Kopie davon ist genauso gut wie das Original. 

Dass formelle Prozesse für die Kunst nicht fremd sind, zeigt sich auf der anderen Seite durch die Verbindung zwischen der Literatur und der Buchdruckkunst, die als ein früher Triumph der mechanischen Datenverarbeitung betrachtet werden darf. Mit Voranbringen dieses Beispiels hoffe ich auch diejenigen zu beruhigen, die sich leicht unwohl bis mulmig mit den modernen Datenautomaten fühlen. Wir alle haben uns längst daran gewöhnt, äußerst persönlich mit Daten umzugehen, die durch mechanische Prozesse erzeugt wurden. Es gibt niemanden, der sich von einem Liebesgedicht, das mit Hilfe von gefühllosen Druckmaschinen reproduziert wurde, beleidigt fühlt. Ab und zu kann man aber beobachten, dass es vielleicht eine Grenze gibt, wie weit wir bereit sind, dieses unpersönlichen Zwischenglied zu akzeptieren. Ich selbst fühle mich beleidigt, wenn ich einen Weihnachtsgruß erhalte, der gedruckt ist und bei dem sich der Absender nicht einmal die Mühe gemacht hat, seine Unterschrift eigenhändig hinzuzufügen. Das Gefühl kann sich aber auch umkehren. Heutzutage wird ein Weihnachtstelegramm ja als vollkommen passende Form des Glückwunsches gezählt, fast besser als ein Brief, obwohl der Telegrammtext, wie man weiß, einer weitaus umfangreicheren Handhabung durch unbefugte Personen und Maschinen ausgesetzt ist. 

Die Verbindung des Datenbegriffs mit dem Modellbegriff ist vielleicht nicht auf den ersten Blick einleuchtend, da man sich unter einem Modell normalerweise ein Ding vorstellt, das in seinem Aussehen eine starke Ähnlichkeit mit dem hat, wovon es ein Modell ist, wie zum Beispiel ein Modellflugzeug. Der Modellbegriff wurde jedoch längst in die empirischen Wissenschaften übernommen, die Physik, die Astronomie, die Biologie, und viele andere, und wird dort für viel abstraktere Konstruktionen gebraucht. In der Astronomie spricht man zum Beispiel von Sternmodellen, die aus Kurven oder Tabellen, also Daten, bestehen. 
}

\transSubSec{Data som værktøj}{Daten als Werkzeug}

\trans{
Men hvis databegrebet på alle disse måder er foregrebet eller indeholdt 1 disse kendte begreber, hvad bringer det da af nyt, må vi spørge. Svaret herpå ligger allerede i det sagte, på den ene side i den ubegrænsede frihed til som data at vælge hvilken art fænomener som helst, lyde, tryksværte på papir, elektriske strømme, magnetfelter, osv., dels i interessen for processerne. I denne sidste interesse ligger noget aktivt. Data er ikke noget der eksisterer eller som man har, det er noget man bruger, et værktøj for mennesker i deres virksomhed. 

Denne opfattelse kan umiddelbart bekræftes ved en overvejelse af sproget. For så vidt sproget anvendes til kommunikation af fakta og ideer mellem mennesker, og ikke som medium for fri fabuleren, da tjener sproget åbenbart som hjælpemiddel, værktøj, — og er vel endda det vigtigste menneskelige værktøj   overhovedet. 

Men med en helt fri opfattelse af data åbner vi vejen for en langt mere effektiv udnyttelse af dem. Sagen er at den lethed hvormed vi kan gennemføre en given dataproces i høj grad afhænger af præcis hvilke data vi har valgt til at repræsentere virkeligheden, eller som jeg kort vil sige, af datarepræsentationen. Dette forhold er velkendt fra sædvanlig brug af de sproglige repræsentationer, tale og skrift. Blandt de processer der her er tale om kan nævnes adskillige, for eksempel: at udveksle ideer blandt personer; at give en fuldstændig beskrivelse af et forhold; at registrere nu tilgængelige fakta for at undgå at de fortabes; at udbrede en enkelt persons ideer til mange. Som enhver ved vælger man sin sproglige form, tale eller skrift, under hensyn til hvorledes processen vil forløbe mest effektivt. Til at udveksle ideer bruger man helst samtale, fordi den skriftlige form er for tung og langsom. Fuldstændige beskrivelser udformes bedst skriftligt, fordi dette er mere fordelagtigt for den der skal bruge dem. Til at registere tilgængelige fakta bruger mange af os skriftlige notater, men det afhænger af vor indbyggede procesformåen, og nogle af os opnår en sikrere registrering ved at bruge hukommelsen. Til at udbrede en enkelt persons ideer til mange benytter man i ikke helt simple tilfælde en kombination af tale og skrift, som vi for eksempel stadig kan iagttage det i arbejdet for at udbrede politiske ideer. 

Den helt fri opfattelse af data har i praksis for længst slået sig igennem hvad angår den dataproces der hedder kommunikation. Vi benytter frit en blanding af de gammelkendte repræsentationer, tale og skrift, og en mangfoldighed af andre, telegrafi, telefoni, radio- og fjernsynsoverføring, idet vi i hver livssituation vurderer den indbyrdes fordel af disse i forhold til den kommunikationsproces vi har brug for at få udført. 

På området datalagring, som kan opfattes som den proces at kommunikere med en forsinkelse af ubekendt varighed, har en voksende frigørelse fra sædvanlig nedskrivning ligeledes fundet sted med fremkomsten af båndoptagere. 

Fra datalogiens synspunkt må denne udvikling opfattes blot som en begyndelse, da den kun berører de allersimpleste dataprocesser. Fremkomsten af datamaterne betyder at langt mere komplicerede dataprocesser meget effektivt kan udføres uden at mennesker behøver at deltage. Vejen er derved blevet åbnet for at arbejdet med datamodeller bliver i høj grad automatiseret og effektiviseret. 
}{
Aber wir dürfen fragen, was neues dieser Datenbegriff bringt, wenn er auf all diese Weisen vorweggenommen oder in diesen bekannten Begriffen bereits enthalten ist. Die Antwort liegt hierbei im bereits Gesagten: auf der einen Seite in der unbegrenzten Freiheit, beliebige Phänomene, Töne, Druckerschwärze auf Papier, elektrische Ströme, Magnetfelder, und viele weiteren, als Daten zu wählen -- und andererseits in den Fokus an den Prozessen. In diesem letzten Fokus liegt etwas aktives. Daten sind nicht etwas, das existiert oder das man hat -- sie sind etwas, das man gebraucht, ein Werkzeug für die Menschen in ihrer Tätigkeit. 

Diese Auffassung lässt sich unmittelbar durch die Berücksichtigung von Sprache. Sofern die Sprache für die Kommunikation von Tatsachen und Ideen zwischen Menschen angewendet wird, und nicht als Medium für freie Fantasien, dann dient die Sprache offensichtlich als Hilfsmittel, als Werkzeug - und ist sogar das wichtigste menschliche Werkzeug überhaupt. 

Aber mit einer völlig freien Auffassung von Daten öffnen wir Wege für eine viel effektivere Verwendung von diesen. Der Fall ist, dass die Leichtigkeit mit der wir einen gegebenen Datenprozess durchführen können, im hohen Grad abhängig davon abhängig ist, welche Daten genau wir gewählt haben, um die Wirklichkeit darzustellen, oder wie ich kurz sagen will, von \emph{Datendarstellungen}. Dieser Zusammenhang ist aus dem üblichen Gebrauch von sprachlichen Darstellungen, der Sprache und der Schrift, gut bekannt. Zu den Prozessen, über die wir hier sprechen, können mehrere als Beispiel  genannt: der Austausch von Ideen zwischen Personen; die vollständige Beschreibung eines Zusammenhangs; die Aufzeichnung von jetzt verfügbaren Fakten, um zu verhindern dass diese verloren gehen; die Verbreitung der Ideen einer einzelnen Person an viele. Wie jeder weiß, wählt man seine sprachliche Form, Sprache oder Schrift, unter der Berücksichtigung wie der Prozess am effektivsten abläuft. Für den Austausch von Ideen benutzt man am liebsten das Gespräch, weil die schriftliche Form zu schwerfällig und langsam ist. Vollständige Beschreibungen lassen sich am besten schriftlich ausformulieren, da dies für denjenigen, der diese benötigt, vorteilhafter ist. Um verfügbare Fakten aufzuzeichen, verwenden viele von uns schriftliche Notizen, aber das hängt von unserer angeborenen Verarbeitungsfähigkeit ab, und einige von uns erreichen eine zuverlässigere Aufzeichnung, wenn sie die Erinnerung benutzen. Um die Ideen einer einzelnen Person an viele Benutzer weiterzugeben, wird in nicht ganz so einfachen Fällen eine Kombination aus Sprache und Schrift verwendet, wie wir zum Beispiel bei den Arbeit zur Verbreitung politischer Ideen immer noch beobachten können.

Die völlig freie Auffassung von Daten hat sich in der Praxis längst durchgesetzt, wenn es um den Datenprozess geht, der Kommunikation heißt. Wir benutzen frei eine Mischung aus altbekannten Darstellungen, Sprache und Schrift, und eine Vielzahl von anderen, Telegrafie, Telefonie, Radio- und Fernsehübertragung, wobei wir in jeder Lebenssituation den jeweiligen Vorteil dieser, in Bezug auf den Kommunikationsprozess den wir fertig stellen wollen, beurteilen. 

Im Bereich der Datenspeicherung, die man als Kommunikation mit einer Verzögerung von unbekannter Dauer auffassen kann, fand mit dem Aufkommen der Tonbandaufnahme ebenfalls eine wachsende Freiten von der üblichen Niederschrift statt.

Vom Standpunkt der \alt{Datalogie}{Informatik} darf man die Auswirkungen dieser Entwicklungen nur als ein Anfang auffassen, da sie bisher nur die einfachsten Datenprozesse betrifft. Die Entstehung von Datenautomaten bedeutet, dass viel kompliziertere Datenprozesse sehr effektiv durchgeführt werden können, ohne dass menschliches Verhalten beteiligt ist.  Dadurch ist der Weg dafür geöffnet, dass die Arbeit mit Datenmodellen im hohen Grad automatisiert und effektiver wird.
}
\transSubSec{Datamodeller i samfundslivet}{Datenmodelle im gesellschaftlichen Leben}

\trans{
Betydningen af datamodeller for samfundslivet kan næppe overvurderes. De er grundlaget for enhver planlægning, fra forretningsmandens tabeller og kurver over den erhvervsmæssige udvikling, over ingeniørens arbejde med beregninger og tegninger der beskriver den bro han er ved at konstruere, til fysikerens matematiske teori for atomkernen. På hvert af disse områder tillader arbejdet med data at man skaffer sig viden om hvordan den virkelige verden vil opføre sig under forskellige omstændigheder. Datamodellerne tillader at vi eksperimenterer med virkeligheden, men uden at vi behøver opføre bekostelige eksperimentopstillinger og uden at tabe ret meget hvis eksperimentet mislykkes. Værdien heraf er især slående ved planlægningen af ting som skibe, flyvemaskiner, broer, og atomreaktorer, hvor et fejlslagent virkeligt eksperiment øjensynlig kan være katastrofalt. Derimod betyder det ikke alverden om det ved en beregning viser sig at en påtænkt reaktorkonstruktion vil eksplodere. 

Hvad er nu betydningen af det datalogiske synspunkt for denne brug af modeller? Her er der to ting at nævne. For det første befrier datalogien os for fordomme om at arbejdet med en bestemt, given problemstilling er knyttet til en bestemt datarepræsentation. Sådanne fordomme er uhyre udbredte. For eksempel arbejder ingeniører og arkitekter traditionelt med tegninger af deres konstruktioner som primære data. Det vil umiddelbart vække modstand hvis man foreslår at denne form måske slet ikke bør være enerådende. Ikke desto mindre er det en kendsgerning at der nu til dags i betydeligt omfang oregår udviklingen af komplicerede konstruktioner, som vejanlæg og skibe, uden nævneværdig brug af tegninger. 

Hermed er ikke sagt at den hidtidige brug af tegninger har været en datalogisk fejltagelse. Hvad datalogien siger er at datarepræsentationen må afpasses efter den transformation der skal udføres og det procesværktøj der er til rådighed. Indtil de moderne datamaters fremkomst omkring 1945 var arkitekter og ingeniører i det væsentlige henvist til at arbejde kun med mennesker som procesværktøj. I relation til den måde vi mennesker arbejder med hovedet og til vor synsmekanik er tegninger en særdeles velvalgt datarepræsentation. 

Den anden værdi af datalogien for odelbyggeren ligger i selve opbygningen af en datamodel og i beskrivelsen af alle detailler i de processer der indgår i dens brug. Den nøje formulering af en datamodel virker i høj grad stimulerende for forståelsen af enhver art problem, stimulerende på en måde som må opleves for at forstås. Denne dybere forståelse af problemerne er slet ikke knyttet kun til mere komplicerede problemer. En betydelig del af gevinsten kan også høstes uden at man kommer i berøring med datamater. Det afgørende er oplevelsen af hvordan dataprocesser kan udvikles, beskrives og bearbejdes

Det fremgår af disse betragtninger at datalogien ikke har nogen nødvendig forbindelse med datamaterne og at den datalogiske betragtningsmåde indeholder meget gammelkendt, som blot ikke hidtil har været samlet under samme synspunkt. Datamaterne har dog det med sagen at gøre at de ved deres fremkomst fremtvinger en ny og velfunderet datalogisk stillingtagen til problemernes løsning. Den hurtighed, sikkerhed og prisbillighed hvormed nutidens datamater gennemfører komplicerede dataprocesser betyder et så afgørende spring sammenlignet med hvad der hidtil har været kendt, at enhver menneskelig aktivitet der i nævneværdig grad har brug for dataprocesser må blive påvirket deraf. Vanskeligheden ved overgangen beror i vid udstrækning på at dyrkerne af de mange specialfag slet ikke er vant til at se deres fag på den led som situationen kræver. Den ingeniør der bygger broer er vant til at vurdere konsekvenserne af nye materialer og nye konstruktive ideer, men ikke i nær så høj grad af nye beskrivelsesformer, nye datarepræsentationer. Samtidig griber en ændring i den datarepræsentation der benyttes under projekteringen så dybt ind i arbejdet at den ikke kan overlades til fremmede eksperter alene, men kræver meget aktiv medvirken af det enkelte fags dyrkere. 

Dertil kommer at datamaterne selv undergår en udvikling, således at tilpasningen af de enkelte fags problemer til dette værktøj ingenlunde er noget der kan klares én gang for alle.
}{
Die Bedeutung von Datenmodellen für das gesellschaftliche Leben kann kaum überschätzt werden. Sie sind die Grundlage für alle Planungen, von den Tabellen und Kurven des Geschäftsmannes über die berufliche Entwicklung, zur Arbeit eines Ingenieurs mit Berechnungen und Zeichnungen die die Brücke die er konstruiert, bis zur mathematischen Theorien des Physikers über den Atomkern. In jedem dieser Bereiche ermöglicht uns die Arbeit mit Daten, dass man sich Wissen darüber schafft, wie die wirkliche Welt sich unter Verschiedenen Umständen verhält. Die Datenmodellierung erlaubt es uns, mit der Realität zu experimentieren, ohne dass wir kostspielige Experimentieraufstellungen aufbauen müssen und ohne dass wir recht viel verlieren, wenn das Experiment missglückt. Der Wert hiervon ist besonders auffällig bei der Planung von Dingen wie Schiffe, Flugzeuge, Brücken, oder Atomreaktoren, wo ein Fehlschlag echter Experimente offensichtlich katastrophal sein kann. Darüber hinaus bedeutet das, dass nicht die ganze Welt darüber Bescheid weiß, wenn eine Berechnung zeigt, dass ein angedachtes Reaktordesign explodieren wird. 

Was ist nun die Bedeutung von diesem \alt{datalogischem}{informatischen} Standpunkt für diesen Gebrauch von Modellen? Hier sind zwei Dinge zu benennen. Erstens befreit uns die \alt{Datalogie}{Informatik} von dem Vorurteil, dass die Arbeit mit einer bestimmten, gegebenen Problemstellung an eine bestimmte Datendarstellung gebunden ist. Solche Vorurteile sind ungeheuer weit verbreitet. So arbeiten zum Beispiel Ingenieure und Architekten traditionell mit Zeichnungen ihrer Entwürfe als primäre Daten. Es wird unmittelbar Widerstand wecken, falls man vorschlägt, dass diese Form vielleicht gar nicht alleine bestimmend sein sollte. Nicht desto weniger ist es eine Tatsache, dass heutzutage in erheblichem Umfang Entwicklungen von komplizierte Konstruktionen, wie Straßenbau und Schiffe, voranschreiten, ohne nennenswerten Gebrauch von Zeichnungen. 

Hiermit wird nicht gesagt, dass die bisherige Verwendung von Zeichnungen ein \alt{datalogischer}{informatischer} Fehler ist. Was die \alt{Datalogie}{Informatik} sagt ist, dass die Datendarstellung an die Transformation die durchgeführt werden soll und das Prozesswerkzeug, das zur Verfügung steht, angepasst werden muss. Bis zum Aufkommen moderner Datenautomaten etwa um 1945 waren Architekten und Ingenieure im Wesentlichen darauf angewiesen, nur mit Menschen als Prozesswerkzeuge zu arbeiten. In Bezug auf die Art und Weise, wie wir Menschen mit dem Kopf und für unserer Sehmechanik arbeiten, sind Zeichnungen eine besonders gut gewählte Datendarstellung. 

Der zweite Wert der \alt{Datalogie}{Informatik} für Modellbauer liegt in der Struktur eines Datenmodell selbst und in der Beschreibung aller Details der Prozesse, die mit seiner Nutzung einhergehen. Die sorgfältige Formulierung eines Datenmodells wirkt im hohen Grade stimulierend für das Verständnis von einem Problem jeder Art -- stimulierend auf eine Art und Weise die man erfahren muss um sie zu verstehen. Dieses tiefere Verständnis der Probleme ist gar nicht nur verbunden mit komplizierteren Problemen. Ein bedeutender Teil des Gewinns kann auch geerntet werden, ohne dass man mit Datenautomaten in Berührung kommt. 
Das Entscheidende ist die Erfahrung, wie Datenprozesse entwickelt, beschrieben und verarbeitet werden.

Es folgt aus diesen Betrachtungen, dass \alt{Datalogie}{Informatik} nicht notwendigerweise eine Verbindung mit Datenautomaten hat und dass die \alt{datalogische}{informatische} Betrachtungsweisen viel bekanntes beinhaltet, das bis jetzt nur noch nicht unter diesen Standpunkt gesammelt wurden. Datenautomaten haben jedoch durch ihr Erscheinen einen neuen und gut begründeten \alt{datalogischen}{informatischen} Standpunkt zur Lösung von Problemen erzwungen. Die Schnelligkeit, Sicherheit und Bezahlbarkeit mit der Datenautomaten heutzutage komplizierte Datenprozesse verarbeiten bedeutet einen so essentiellen Sprung, verglichen mit was wir bis hierher gekannt hatten, dass jede menschliche Aktivität die im nennenswerten Grad Datenprozesse braucht, davon betroffen sein wird. Die Schwierigkeit des Übergangs betrifft zu einem großen Teil die Beschäftigten von den vielen Spezialfächern, die es gar nicht gewohnt sind, ihr Fach auf diese Weise, wie es die Situation erfordert, zu sehen. Der Ingenieur, der Brücken baut, ist es gewohnt, die Folgen neuer Materialien und neuer Konstruktionsideen zu bewerten, aber bei weitem nicht so sehr, die neue Beschreibungsformen oder neuer Datendarstellungen. Gleichzeitig greift eine Änderung in die Datendarstellungen, die bei der Planung benutzt werden, so tief in die Arbeit ein, dass diese nicht fremden Experten alleine überlassen werden kann, sondern eine sehr aktive Mitwirkung von den einzelnen Fächern benötigt.

Dazu kommt, dass Datenautomaten selbst eine Entwicklung durchlaufen, so dass die Anpassung auf die Probleme der einzelnen Fächer zu diesen Werkzeugen gar nicht etwas ist, das ein für alle Mal erledigt werden kann.
}

\transSubSec{Datalogi i uddannelsen}{\alt{Datalogie}{Informatik} in Ausbildungen}

\trans{
Har man således indset at datalogien på den ene side sammenfatter en lang række centrale menneskelige aktiviteter og begrebsdannelser under ét samlende synspunkt, og på den anden side formår at befrugte og forny tankegangen i en lige så lang række fag, da kan man ikke være i tvivl om at datalogien må have en plads i almenuddannelsen. For at nå til en rimelig forestilling om hvordan denne placering bør være er det naturligt at sammenligne med fag af lignende karakter. Man vil da nå frem til sproglære og matematik, som er de nærmeste analoge. Både datalogien og disse to fagområder beskæftiger sig med tegn og symboler der er opfundet af mennesker som hjælpemidler. Fælles for de tre emner er også deres karakter af redskaber for mange andre fag. I uddannelsen må de derfor indgå på to måder, dels som hjælpefag ved studier af mange andre fag, dels som hovedfag ved uddannelsen af specialister i selve disse emner. Vi har jo alle måttet gennemgå meget betydelige mængder sprog, regning og matematik i skolen, uanset at kun ganske få af os er blevet lingvister eller matematikere. På lignende måde må datalogien bringes ind i skoleundervisningen og forberede os alle på tilværelsen i datamaternes tidsalder, ganske som læsning og skrivning anses som en nødvendig forudsætning for tilværelsen i et samfund der er præget af tryksager. 

Datalogien kan tænkes placeret i skoleundervisningen, enten som et selvstændigt fag, eller som en del af matematikken. Det afgørende er indholdet af undervisningen. Hovedtemaerne må være data, datarepræsentationer, og dataprocesser. Disse fundamentale begreber må belyses gennem en række konkrete eksempler som bekvemt kan tages fra områder inden for skoleelevernes erfaringsområde, talregning, stavning, opslag i registre, og lignende. Der bør også indgå simple eksempler på brugen af mekaniske og elektriske fænomener til datarepræsentation. Disse eksempler bør belyses ved simple forsøg. Datamaterne bør også omtales, men ikke som det centrale i faget, snarere som en afsluttende orientering.  

Når datalogien er blevet vel etableret i almen uddannelsen vil den mystik der omgiver datamaterne i manges forestillinger opløse sig i intet. Dette må betragtes som den måske allervigtigste begrundelse for at fremme forståelsen af datalogien. Det vil nemlig være betingelsen for at herredømmet over datamaterne og deres anvendelse ikke bliver en sag for en lille gruppe af eksperter, men bliver en sædvanlig politisk sag, og således gennem det politiske system kommer til at ligge hvor det bør, hos os alle.
}{
Hat man also erkannt, dass \alt{Datalogie}{Informatik} auf der einen Seite eine lange Reihe zentraler menschlicher Aktivitäten und Begriffsbildungen unter einen gemeinsamen Standpunkt zusammenfasst, und dass sie auf der anderen Seite fähig ist, in einer genauso langen Reihe Fächer die Denkweise zu befruchten und zu erneuern, so kann man nicht im Zweifel darüber bleiben, dass \alt{Datalogie}{Informatik} einen Platz in der Allgemeinbildung haben muss. Um eine vernünftige Vorstellung davon zu bekommen, wie dieser Platz sein sollte, ist es natürlich, mit Fächern von ähnlichem Charakter zu vergleichen. Dies führt zu der Sprachlehre und Mathematik, die am ehesten vergleichbar sind. Sowohl die \alt{Datalogie}{Informatik} als auch diese beiden Themenbereiche beschäftigen sich mit Zeichen und Symbolen, die von Menschen als Hilfsmittel erfunden wurden. Gleich für diese drei Fächer ist auch deren Charakter als Werkzeuge für viele andere Fächer. In der Ausbildung müssen sie daher auf zwei Arten eingehen: teils als Hilfsfächer für das Studieren von vielen anderen Fächern, teils als Hauptfächer bei der Ausbildung von Spezialisten in diesen Fächern selbst. Wir haben ja schließlich alle in großen Umfang Sprache, Rechnung und Mathematik in der Schule gelernt, auch wenn nur ganz wenige von uns Linguisten oder Mathematiker geworden sind. In ähnlicher Weise muss die \alt{Datalogie}{Informatik} in die Schullehre gebracht werden und uns alle auf die Existenz im Zeitalter der Datenautomaten vorbereiten, ganz so wie Lesen und Schreiben als notwendige Voraussetzung für die Existenz in einer Gesellschaft, die von Drucksachen geprägt ist, angesehen werden.

\alt{Datalogie}{Informatik} kann im Schulunterricht entweder als eigenständiges Fach, oder als ein Teil der Mathematik, gedacht werden. Entscheidend ist der Inhalt des Unterrichts. Die Hauptthemen sollten Daten, Datendarstellungen und Datenprozesse sein. Diese fundamentalen Begriffe müssen durch eine Reihe von konkreten Beispielen veranschaulicht werden, die bequem aus dem Erfahrungsbereich der Schüler genommen werden können -- wie Zahlrechnung, Rechtschreibung, Nachschlagen in Registern, und dergleichen. Es sollten auch einfache Beispiele für den Gebrauch von mechanischen und elektrischen Phänomene zur Datendarstellung enthalten sein. Diese Beispiele sollten durch einfache Versuche veranschaulicht werden. Datenautomaten sollten auch erwähnt werden, aber nicht als Kernstück des Faches, sondern eher als abschließende Orientierung.  

Wenn die \alt{Datalogie}{Informatik} also erst einmal in der Allgemeinbildung etabliert ist, wird sich die Mystik, die die Datenautomaten in vielen Vorstellungen umgibt, auflösen. Dies darf vielleicht als die allerwichtigste Begründung für die Förderung des Verständnisses von \alt{Datalogie}{Informatik} verstanden werden. Dies wird nämlich die Voraussetzung dafür sein, dass die Herrschaft über Datenautomaten und deren Anwendung nicht eine Sache von einer kleinen Gruppe von Experten bleibt, sondern eine übliche politische Sache wird, und somit durch das politische System dorthin gelangt, wo sie sein sollte, nämlich bei uns allen.
}
