\transSec{DATAMATERNES ANVENDELSE I DAG}{Die heutige Anwendung von Datenautomaten}

\trans{
I dag skal vi gennemgå nogle anvendelser af datamaterne, sådan som de finder sted ved de talrige datamater der er spredt ud over kloden. En sådan gennemgang inden for rammerne af denne forelæsning må nødvendigvis blive stærkt selektiv. De udvalg jeg har foretaget har delvis været ledet af ønsket om at give endnu et aspekt af svaret på hvorfor datamaterne er så samfundsmæssigt betydningsfulde. Jeg har derfor valgt en række anvendelser af forholdsvis jordnær art, som er forholdsvis lette at forstå hvis man blot har en smule kendskab til de pågældende aktiviteter. Disse eksempler har jeg suppleret med anvendelser som ikke er så indlysende, men som man kan belyse nogle af de metoder der ligger bag anvendelserne.
}{
Heute werden wir einige Anwendungen der Datenautomaten durchgehen, wie sie bei den zahlreichen Datenautomaten stattfinden, die über den ganzen Erdball verteilt sind. Eine solche Durchsicht im Rahmen dieser Vorlesung muss notwendigerweise stark selektiv sein. Die Auswahl, die ich getroffen habe, wurde teilweise von dem Wunsch geleitet, noch einen Aspekt zur Antwort, warum die Datenautomaten so gesellschaftlich bedeutsam sind, hinzuzufügen. Ich habe daher eine Reihe von relativ bodenständigen Anwendungen ausgewählt, die relativ leicht zu verstehen sind, wenn man nur ein wenig Kenntnis der zugehörigen Tätigkeiten hat. Diese Beispiele habe ich durch Anwendungen ergänzt, die nicht so offensichtlich sind, bei denen man jedoch einige der Methoden beleuchten kann, die hinter den Anwendungen liegen.
}

\transSubSec{Informationssøgning}{Informationssuche}

\trans{
Jeg vil begynde med nogle anvendelser af datamater til at håndtere tekster fra sædvanligt sprog. Herigennem får jeg atter engang lejlighed til at betone at vi allerede for længst er ude over det stadium da beregninger var det vigtigste arbejdsområde for dem. Vi skal herunder tale dels om informationssøgning, dels om sprogoversættelse. 

Informationssøgning er blandt andet af interesse ved den videnskabelige litteratur, der som bekendt er vokset enormt i omfang i de senere år. For mange fag gælder det at den mængde faglitteratur der fremkommer i tidsskrifter er så stor at den enkelte forsker er ude af stand til blot at se det igennem for at vide om der er fremkommet nyheder af interesse for ham. Her kan datamaterne hjælpe. Dertil kræves at den pågældende litteratur, enten den fuldstændige tekst eller titlen og passende beskrivende nøgleord, overføres til et medium som kan læses af datamaten, for eksempel magnetbånd. Man har nu lavet et program for en datamat der efter at have fået opgivet visse søgeord gennemsøger hele denne tekst og som nårsomhelst det i teksten støder på et af søgeordene sørger for at aflevere oplysninger om den videre sammenhæng på dette sted i teksten, i særdeleshed hvad det er for en afhandling det stammer fra. Opgaven for den forsker der vil gøre brug af dette program er at angive de ord programmet skal bruge som søgeord. Man vil forstå at systemet ikke er betinget af nogen klassifikation af litteraturen. Det vil derfor uden vanskelighed kunne behandle stadig nye ord og emner. Tillige vil brugeren uden videre som søgeord kunne angive højt specialiserede ord, sære tekniske betegnelser, og vil derved fra datamaten få et meget begrænset svar, koncentreret om netop det han har spurgt om. 

Blandt videnskaberne er det først og fremmest kemien der lider under den enorme vækst i litteraturen. På dette område er den datamatiske litteratursøgning efter de omtalte principper blevet taget op i stor skala i U.S.A. Den primære, besværlige overføring af titler for alle nye kemiske afhandlinger udføres løbende af en enkelt central institution og er derefter til rådighed for alle andre mod passende vederlag. Materialet er herhjemme tilgængeligt gennem Danmarks tekniske Bibliotek, der som det første i Europa har sluttet sig til dette arbejde. 
}{
Ich möchte mit einigen Anwendungen der Datenautomaten beginnen, um Texte in gewöhnlicher Sprache zu verarbeiten. Dadurch bekomme ich erneut die Gelegenheit zu betonen, dass wir das Stadium längst hinter uns gelassen haben, in dem Berechnungen ihr wichtigstes Arbeitsfeld waren. Wir werden dabei teilweise über Informationssuche und teilweise über Sprachübersetzung sprechen.

Die Informationssuche ist unter anderem von Interesse bei der wissenschaftlichen Literatur, die, wie bekannt, in den letzten Jahren enorm an Umfang zugenommen hat. Für viele Fächer gilt, dass die Menge an Fachliteratur, die in Zeitschriften erscheint, so groß ist, dass der einzelne Forscher außerstande ist, sie nur zu überblicken, um zu wissen, ob Neuigkeiten von Interesse für ihn aufgetaucht sind. Hier können die Datenautomaten helfen. Dazu ist es erforderlich, dass die betreffende Literatur, entweder der vollständige Text oder der Titel und passende beschreibende Schlüsselwörter, auf ein Medium übertragen wird, das vom Datenautomaten gelesen werden kann, wie zum Beispiel ein Magnetband. Man hat nun ein Programm für einen Datenautomaten erstellt, das, nachdem es bestimmte Suchwörter erhalten hat, den gesamten Text durchsucht und wann immer es im Text auf eines der Suchwörter stößt, dafür sorgt, Auskunft über den weiteren Zusammenhang an dieser Stelle im Text zu liefern, insbesondere, aus welcher Abhandlung es stammt. Die Aufgabe des Forschers, der dieses Programm nutzen möchte, besteht darin, die Wörter anzugeben, die das Programm als Suchwörter verwenden soll. Man wird verstehen, dass das System nicht von einer (bestimmten) Klassifikation der Literatur abhängt. Es wird daher ohne Schwierigkeiten in der Lage sein, ständig neue Wörter und Themen zu verarbeiten. Außerdem wird der Benutzer ohne Weiteres hoch spezialisierte Wörter oder seltsame technische Begriffe als Suchwörter angeben können, und wird dadurch vom Datenautomaten eine sehr begrenzte Antwort erhalten, die konzentriert auf genau das ist, wonach er gefragt hat.

Unter den Wissenschaften ist es vor allem die Chemie, die unter dem enormen Wachstum der Literatur leidet. In den USA wurde (bereits) in diesem Bereich die datenautomatengestützte Literatursuche nach den genannten Prinzipien in großem Maßstab aufgenommen. Die primäre, mühsame Übertragung der Titel aller neuen chemischen Abhandlungen wird fortlaufend von einer einzigen zentralen Institution durchgeführt und steht dann allen anderen gegen eine entsprechende Gebühr zur Verfügung. Das Material ist hierzulande durch die Dänische technische Bibliothek $^{(5.1)}$
zugänglich, die sich als erste in Europa diesem Projekt angeschlossen hat.

}


\transSubSec{Sprogoversættelse}{Sprachübersetzung}

\trans{
En anden sprogmæssig opgave, der har været genstand for megen interesse, er oversættelse. Denne interesse kan ikke undre i vor tid, med det store behov for kommunikation tværs over sproggrænserne. Det kan også umiddelbart forekomme at være en nærliggende opgave for en datamat. Både det der går ind og det der kommer ud af en oversættelse er jo tekster, altså data, og kan derfor umiddelbart håndteres i datamaterne. Opgaven har dog vist sig at være langt vanskeligere end antaget midt i 1950'erne da problemet blev taget op mange steder i en rus af optimisme. På denne tid kneb det endnu stærkt med datamaternes kapacitet for data og det at levere datamaten et leksikon på en tilgængelig form strakte mulighederne til det yderste. De der arbejdede med oversættelsesproblemet kunne derfor endnu forestille sig at når blot dette kapacitetsproblem blev overvundet så var hele oversættelsesopgaven løst. Da man så fik langt større datakapaciteter til rådighed i datamaterne viste dette sig at være en illusion. Det er blevet mere og mere klart at en kvalitetsoversættelse kræver en analyse af den givne tekst der medtager sammenhæng ikke alene inden for den enkelte sætning, men fra sætning til sætning og fra paragraf til paragraf. Dette har hidtil ikke kunnet realiseres og kyndige inden for feltet er forberedt på at det vil tage mange år endnu før problemet er løst. 

Dette modsiges ikke af at der adskillige steder i verden har været demonstreret oversætterprogrammer for offentligheden. Oversættelse er ikke noget man kan eller ikke kan, men i høj grad et kvalitetsspørgsmål. De oversættere der har været demonstreret har alle været ret primitive. Dermed er ikke sagt at de ikke kan være nyttige til visse formål, for eksempel oversættelse af videnskabelig eller teknisk litteratur. Hvad der for øjeblikket ligger helt uden for disse oversætterprogrammers rækkevidde er skønlitterære oversættelser. Ved god skønlitteratur ligger en væsentlig del af værkets kvalitet i selve det sproglige udtryk og oversættelsen bliver en vanskelig kunst, i følge Karen Blixen den sværeste af alle. Hvor det drejer sig om en oversættelse der er betinget af en helhedsopfattelse af en tekst er mennesker stadig datamaterne langt overlegne. 
}{
Eine andere sprachliche Aufgabe, die Gegenstand großen Interesses geworden ist, ist die Übersetzung. Dieses Interesse überrascht in unserer Zeit aufgrund des großen Bedarfs an Kommunikation über Sprachgrenzen hinweg nicht. Die Sprachübersetzung mag auch unmittelbar als eine naheliegende Aufgabe für einen Datenautomaten erscheinen. Sowohl das, was in eine Übersetzung hineingeht, als auch das, was herauskommt, sind ja Texte, also Daten, und können daher direkt in den Datenautomaten verarbeitet werden. Die Aufgabe hat sich jedoch als weitaus schwieriger erwiesen als Mitte der 1950er Jahre angenommen, als das Problem vielerorts in einem Rausch von Optimismus angegangen wurde. Zu dieser Zeit war die Kapazität der Datenautomaten für Daten noch stark begrenzt, und dem Datenautomaten ein Lexikon in einer zugänglichen Form bereitzustellen, reizte die Möglichkeiten bis zum Äußersten aus. Diejenigen, die am Übersetzungsproblem arbeiteten, konnten sich daher noch vorstellen, dass die gesamte Übersetzungsaufgabe gelöst sei, sobald dieses Kapazitätsproblem überwunden war. Als man dann viel größere Datenkapazitäten in den Datenautomaten zur Verfügung hatte, erwies sich dies als Illusion. Es wurde immer deutlicher, dass eine qualitativ hochwertige Übersetzung eine Analyse des gegebenen Textes nicht nur erfordert, die Zusammenhänge innerhalb des einzelnen Satzes zu berücksichtigen, sondern auch die von Satz zu Satz und von Absatz zu Absatz. Dies konnte bisher nicht realisiert werden, und Experten auf diesem Gebiet erwarten, dass es noch viele Jahre dauern wird, bis das Problem gelöst ist.

Dies wird nicht dadurch widerlegt, dass an mehreren Orten weltweit Übersetzungsprogramme für die Öffentlichkeit demonstriert wurden. Übersetzung ist nicht etwas, das man entweder kann oder nicht kann -- sondern in hohem Maße eine Frage der Qualität. Die Übersetzungsprogramme, die demonstriert wurden, waren alle ziemlich primitiv. Damit ist nicht gemeint, dass sie nicht für bestimmte Zwecke nützlich sein können -- wie zum Beispiel für die Übersetzung von wissenschaftlicher oder technischer Literatur. Was jedoch im Moment völlig außerhalb der Reichweite dieser Übersetzungsprogramme liegt, sind belletristische Übersetzungen. Bei guter Belletristik liegt ein wesentlicher Teil der Qualität des Werks im sprachlichen Ausdruck selbst, und die Übersetzung wird zu einer schwierigen Kunst -- laut Karen Blixen $^{(5.2)}$ sogar die schwierigste von allen. Wenn es um eine Übersetzung geht, die auf einem ganzheitlichen Verständnis eines Textes beruht, sind Menschen den Datenautomaten immer noch weit überlegen.
}

\transSubSec{Sproganalyse}{Sprachanalyse}

\trans{
Datamaternes styrke overfor tekster i naturligt sprog ligger i deres detailanalyse af enkelte sætninger. Sådanne analyser kan være af betydelig interesse ved tekster hvis nøjagtige udlægning indebærer betydningsfulde konsekvenser, i særdeleshed lovtekster. Som et interessant eksempel på en sådan anvendelse kan nævnes en analyse af visse centrale sætninger i den internationale traktat om forbud mod kernevåbenprøver. Disse sætninger blev analyseret af et program der er udviklet ved Harvard universitetet. Resultatet af analysen er oplysninger om hvordan ordene i sætningen grupperes og bestemmer hinanden, og i givet fald hvorledes ordene kan grupperes på mere end én måde. Det interessante ved dette eksempel var at der blandt blot ni sætninger blev fundet én der kunne opfattes på to afgørende forskellige måder. Den pågældende sætning lyder i den originale engelske formulering: 

»It is understood in this connection that the provisions of this subparagraph are without prejudice to the conclusion of a treaty resulting in the permanent banning of all nuclear test explosions, including all such explosions underground.« 

Omtrentligt oversat til dansk lyder sætningen: »Det forstås i denne forbindelse at aftalerne i denne underparagraf ikke stiller sig i vejen for afslutningen af en traktat der resulterer i et permanent forbud mod alle kernevåbenprøveeksplosioner, inklusive alle sådanne eksplosioner under jorden.« 

Det ejendommelige er imidlertid at den originale engelske version kan læses som at »aftalerne i denne underparagraf ikke stiller sig i vejen for en traktat og resulterer i et permanent forbud mod alle kernevåbenprøveeksplosioner, inklusive alle sådanne eksplosioner under jorden«, hvilket modsiger traktatens idé, der som bekendt ikke forbyder underjordiske prøver. Gennem sin pedantiske, systematiske analyse har datamaten altså afsløret en tvetydighed, som er tydelig nok når den først er påvist, men som falder så lidt i øjnene at den uantastet er passeret de trænede juridiske hoveder der har udtænkt teksten. 
}{
Die Stärke der Datenautomaten gegenüber Texten in natürlicher Sprache liegt in ihrer Detailanalyse einzelner Sätze. Solche Analysen können von erheblichem Interesse sein bei Texten, deren genaue Auslegung bedeutungsvolle Konsequenzen hat, insbesondere bei Gesetzestexten. Als interessantes Beispiel für eine solche Anwendung kann eine Analyse bestimmter zentraler Sätze im internationalen Vertrag über das Verbot von Kernwaffentests genannt werden $^{(5.3)}$.
Diese Sätze wurden von einem Programm analysiert, das an der Harvard-Universität entwickelt wurde. Das Ergebnis der Analyse sind Informationen darüber, wie die Wörter im Satz gruppiert und einander zugeordnet werden, und gegebenenfalls, wie die Wörter auf mehr als eine Weise gruppiert werden können. Das Interessante an diesem Beispiel war, dass unter nur neun Sätzen einer gefunden wurde, der auf zwei entscheidend unterschiedliche Weisen interpretiert werden konnte.

%Der betreffende Satz lautet in der originalen englischen Formulierung: It is understood (Es wird verstanden) in this connection (in dieser Verbindung) that the provisions (dass die Provisionen/Vorschriften) of this subparagraph (von diesem Unterparagraph) are without prejudice (ohne vorgefasste Meinung sind / keinen Schaden zufügen an) to the conclusion (zum Abschluss) of a treaty (eines Vertrages) resulting in (der resultiert in) the permanent banning (einem permanenten Verbot) of all nuclear test explosions (von allen Kerntestexplosionen/Nuklearexplosionen), including all such (die auch beinhalten all jene) explosions underground (explosionen untergrund).« 

Der betreffende Satz lautet in der originalen englischen Formulierung: »It is understood in this connection that the provisions of this subparagraph are without prejudice to the conclusion of a treaty resulting in the permanent banning of all nuclear test explosions, including all such explosions underground.« 

Ungefähr ins Dänische übersetzt $^{(5.4)}$ lautet der Satz: Det forstås (Es wird verstanden) i denne forbindelse (in dieser Verbindung / in diesem Zusammenhang) at aftalerne (dass Verabredungen) i denne underparagraf (in diesem Unterparagraph) ikke stiller sig (sich nicht stellen) i vejen (in den Weg) for afslutningen (zum Abschließen) af en traktat (von einem Vertrag) der resulterer (der darin resuliertiert) i et permanent forbud (zu einem permanenten Verbot) mod alle kernevåbenprøveeksplosioner (von allen Kernwaffenexplosionen), inklusive alle sådanne (inklusive aller derartigen) eksplosioner under jorden (explosionen unter der Erde).

Das Merkwürdige ist jedoch, dass die originale englische Version so gelesen werden kann, dass die Bestimmungen in diesem Unterabsatz der Verabschiedung eines Vertrags nicht im Wege stehen und (darüber hinaus) zu einem dauerhaften Verbot aller Kernwaffentestexplosionen führen, einschließlich aller solcher Explosionen unter der Erde -- was der Idee des Vertrags widerspricht, der bekanntlich unterirdische Tests nicht verbietet. Durch seine pedantische, systematische Analyse hat der Datenautomat also eine Zweideutigkeit aufgedeckt, die klar genug ist, sobald sie aufgezeigt wird, die jedoch so wenig auffällt, dass sie unangefochten an den geschulten juristischen Köpfen vorbeigegangen ist, die den Text erdacht haben.
}

\transSubSec{Atomreaktorer}{Atomreaktoren}

\trans{
Jeg vil nu gå over til at omtale en række anvendelser af datamater til hjælp ved arbejdet med datamodeller. Man kan her skelne mellem projekteringsopgaver, som går ud på at opbygge en helt ny konstruktion og hvor arbejdet med datamodellen er overstået når konstruktionen er fuldt ud planlagt, og overvågnings- og styringsopgaver, hvor datamodellen benyttes sideløbende med at den tilsvarende virkelighed udfolder sig.

For at begynde med projekteringsopgaverne, lad os først tænke på et af de felter hvor datamater har været brugt flittigst, nemlig beregninger over atomreaktorer. Reaktorfolkenes stærke interesse for datamaterne skyldes dels at feltet er nyt og i hastig udvikling, dels at man selvsagt er parat til at gøre det ekstraordinære i retning af omhyggelig projektering på grund af dem store samfundsmæssige risiko der er knyttet til atomreaktorer. 

Det man undersøger med datamaternes hjælp er de fysiske forhold i reaktorerne, strømmen af neutroner i reaktorkernen, strålings- og varmeudvekslingen med omgivelserne, ændringerne i reaktorbrænslet med tiden, reaktorens opførsel i 
tilfælde af at forskellige styringsmekanismer svigter, og lignende. Beregningerne baseres på et omfattende kendskab til stoffernes egenskaber, sådan som fysikerne har bragt det til veje. Dertil kommer et omfattende datamateriale der beskriver opbygningen af den bestemte reaktor som man undersøger. Disse omstændigheder gør beregningerne uhyre besværlige og det er en kendsgerning at det gang på gang har været reaktorfolkene der har presset på for at få datamatfabrikanterne til at udvikle større og hurtigere datamater. 
}{
Ich werde nun eine Reihe von Anwendungen der Datenautomaten besprechen, die der Unterstützung bei der Arbeit mit Datenmodellen dienen. Hier kann man zwischen Planungsaufgaben unterscheiden, die darauf abzielen, eine völlig neue Konstruktion zu erstellen und bei denen die Arbeit mit den Datenmodellen abgeschlossen ist, wenn die Konstruktion vollständig geplant ist -- und Überwachungs- und Steuerungsaufgaben, bei denen die Datenmodelle parallel zur sich verändernden Realität verwendet werden.

Beginnen wir mit den Planungsaufgaben und denken zunächst an eines der Gebiete, in denen Datenautomaten am häufigsten verwendet wurden, nämlich Berechnungen für Atomreaktoren. Das starke Interesse der Reaktorfachleute an den Datenautomaten liegt einerseits darin, dass das Feld neu und in rascher Entwicklung ist -- und anderseits, dass man, aufgrund des großen gesellschaftlichen Risikos, das mit Atomreaktoren verbunden ist, selbstverständlich bereit ist, außerordentliches in Richtung sorgfältiger Planung zu unternehmen.

Was man mit Hilfe der Datenautomaten untersucht, sind die physikalischen Bedingungen in den Reaktoren, der Neutronenfluss im Reaktorkern, der Strahlungs- und Wärmeaustausch mit der Umgebung, die Veränderungen des Reaktorbrennstoffs im Laufe der Zeit, das Verhalten des Reaktors im Falle eines Ausfalls verschiedener Steuerungsmechanismen und dergleichen. Die Berechnungen basieren auf umfassendem Wissen über die Eigenschaften der Stoffe, wie es von den Physikern ermittelt wurde. Hinzu kommt umfangreiches Datenmaterial, das den Aufbau des jeweiligen Reaktors beschreibt, den man untersucht. Diese Umstände machen die Berechnungen äußerst schwierig, und es tatsächlich so, dass Reaktorfachleute immer wieder Druck auf die Datenautomatenhersteller ausgeübt haben, um größere und schnellere Datenautomaten zu entwickeln.
}

\transSubSec{Vejanlæg}{Straßenbau}

\trans{
Flere gode eksempler på samfundsmæssigt betydningsfulde projekteringsopgaver er knyttet til vejbygning. En af de mere enkle problemstillinger her er at finde frem til en placering af en vej gennem et kendt terræn således at jordflyttearbejdet bliver reduceret mest muligt. Ved mere ambitiøse projekter tilstræber man at få datamaten til at medtage stadig flere hensyn ved vejplaceringen, for eksempel vejens overskuelighed for trafikanterne. Ved de mest raffinerede projekter af denne art lader man datamaten fremstille trafikantens udsyn som et billede på et katodestrålerørs skærm, og giver vejingeniøren mulighed for at eksperimentere med vejføringen.

En anden problemstilling fra dette felt er tilrettelæggelsen af vejkryds således at man på én gang opnår stor sikkerhed og god trafikkapacitet. Til arbejder af denne art har man ofte ingen anden måde end at prøve sig frem, eller som det kaldes teknisk, at simulere processen. Denne fremgangsmåde ligner børnenes spil med en vejplan, små modelbiler og terningkast. For at få noget nyttigt frem må man blot omhyggeligt fastlægge spillereglerne således at de afspejler den projekterede vejplan, trafiktætheden, og trafikanternes reaktioner. Mens simulationen er i gang vil man med passende data have et billede af vejkrydset og de køretøjer der i et givet øjeblik befinder sig i det, tillige med deres positioner og den fart og retning de kører, samt oplysninger om førernes egenskaber. Man lader nu tiden forløbe med passende små skridt og holder stadig regnskab med køretøjerne og førernes reaktioner. Når for eksempel en bilist bemærker at den foran kørende bil bremser, da vil han selv bremse, men det sker med en vis forsinkelse, som endda ikke er den samme for alle. Denne variation fra den ene bilist til den anden kan man tage hensyn til ved stadig at lade det der sker være noget afhængig af et element af tilfældighed, som ved kast af en terning. Ved at følge en sådan trafikafvikling gennem et langt tidsrum, som tillader et stort antal køretøjer med egenskaber som fordeler sig realistisk over de kendte variationer i reaktionstid, osv. vil man 
kunne få en viden om hvor hurtigt trafikken kan afvikles og hvor tit der vil ske uheld. Ved at gennemføre sådanne undersøgelser for forskelligt udformede kryds vil man kunne finde ud af hvilken udformning der er bedst. 

Der skal imidlertid spilles længe med et sådant trafikspil før man kan stole på resultatet, og her kommer datamaterne ind, for selv om det kan lyde mærkeligt, så kan hele spillet afvikles i en datamat. Datamaten kan rigtig nok ikke kaste med terning, men dette viser sig ikke at være en afgørende vanskelighed, det er muligt i en datamat at frembringe serier af tal der tilstrækkelig tilfældige til formålet. 
}{
Mehrere gute Beispiele für gesellschaftlich bedeutsame Planungsaufgaben sind mit dem Straßenbau verbunden. Eines der einfacheren Probleme hierbei ist es, einen Standort für eine Straße durch bekanntes Gelände zu finden, sodass die Erdbewegungsarbeiten möglichst minimiert werden. Bei ambitionierteren Projekten strebt man an, den Datenautomaten dazu zu bringen, bei der Straßenführung immer mehr Aspekte zu berücksichtigen, zum Beispiel die Übersichtlichkeit der Straße für die Verkehrsteilnehmer. Bei den raffiniertesten Projekten dieser Art lässt man den Datenautomaten das Sichtfeld des Verkehrsteilnehmers als Bild auf einem Kathodenstrahlröhrenbildschirm darstellen und gibt dem Straßenbauingenieur die Möglichkeit, mit der Straßenführung zu experimentieren.

Eine andere Problemstellung aus diesem Bereich ist die Gestaltung von Straßenkreuzungen, sodass man gleichzeitig große Sicherheit und hohe Verkehrskapazität erreicht. Bei Arbeiten dieser Art gibt es oft keine andere Möglichkeit, als es auszuprobieren, oder wie es technisch genannt wird, den Prozess zu simulieren. Diese Vorgehensweise ähnelt dem Spiel von Kindern mit einem Straßenplan, kleinen Modellautos und Würfeln. Um etwas Nützliches zu erzielen, muss man nur sorgfältig die Spielregeln festlegen, sodass sie den geplanten Straßenverlauf, die Verkehrsdichte und die Reaktionen der Verkehrsteilnehmer abbilden. Während die Simulation läuft, wird man mit geeigneten Daten ein Bild der Kreuzung und der Fahrzeuge haben, die sich zu einem bestimmten Zeitpunkt dort befinden, zusammen mit ihren Positionen und der Geschwindigkeit und Richtung, in die sie fahren, sowie Informationen über die Eigenschaften der Fahrer. Man lässt nun die Zeit in geeigneten kleinen Schritten vergehen und verfolgt weiterhin die Fahrzeuge und die Reaktionen der Fahrer. Wenn zum Beispiel ein Autofahrer bemerkt, dass das vorausfahrende Auto bremst, dann wird er selbst bremsen, aber das geschieht mit einer gewissen Verzögerung, die sogar nicht bei allen gleich ist. Diese Variation von einem Autofahrer zum anderen kann man berücksichtigen, indem man das Geschehen weiterhin von einem Element des Zufalls abhängig macht, wie beim Würfeln. Indem man einen solchen Verkehrsfluss über einen langen Zeitraum hinweg verfolgt, der eine große Anzahl von Fahrzeugen mit Eigenschaften enthält, die sich realistisch über die bekannten Schwankungen in der Reaktionszeit verteilen, wird man herausfinden, wie schnell der Verkehr fließen kann und wie oft Unfälle passieren. Indem man solche Untersuchungen für unterschiedlich gestaltete Kreuzungen durchführt, wird man herausfinden können, welche Gestaltung die beste ist.

Man muss jedoch lange mit einem solchen Verkehrsspiel spielen, bevor man dem Ergebnis vertrauen kann und hier kommen die Datenautomaten ins Spiel. Denn auch wenn es merkwürdig klingen mag, kann das ganze Spiel in einem Datenautomaten ablaufen. Der Datenautomat kann zwar keine Würfel werfen, aber das stellt sich nicht als entscheidendes Problem heraus -- es ist möglich, in einem Datenautomaten Zahlenreihen zu erzeugen, die für den Zweck ausreichend zufällig sind.
}

\transSubSec{Penge- og lageradministration}{Geld- und Lagerverwaltung}

\trans{
Lad os nu overveje datamaternes anvendelser til at hjælpe ved overvågningen og styringen af løbende aktiviteter. Først er der hertil at bemærke at en sådan datamæssig overvågning på ingen måde er noget nyt, men i mangfoldige år har fundet sted ved det papir- og kontorarbejde der finder sted i virksomhederne og statsmaskineriet. Det er derfor ikke overraskende at datamaterne først fik indpas i den løbende styring ved at de overtog en del af det papirarbejde der hidtil havde fundet sted. Blandt de første brugere af datamater på dette område var bankerne. Det stof bankerne arbejder med er jo for største delen penge, altså data, lige parat til at bearbejdes med datamat. 

Produktionsvirksomheder frembyder langt mere varierende problemer, og udnyttelsen af højt udviklet datamatik kræver en mere omfattende omstilling. Et af de områder der først blev taget op er lagerregnskabet. Det er først og fremmest påtrængende ved produktioner der består i at samle mange smådele til større apparater, for eksempel radioindustrien. Netop på grund af at apparaterne er opbygget af mange smådele er det fristende for fabrikanten at tilbyde mange forskellige varianter af sine produkter. Det virker jo umiddelbart så enkelt blot at sørge for at visse af de dele der indgår i de færdige apparater kan vælges med en passende variation, for eksempel således at et radioapparat kan leveres til flere forskellige forsyningsspændinger, til brug i forskellige lande. Ved mere omfattende produktioner viser det sig dog at problemet at holde styr på en produktion af denne type hurtigt kan vokse uhyggeligt i omfang. For at produktionen skal holdes jævnt kørende er det nødvendigt at de nødvendige enkeltdele hele tiden bringes frem til montørerne til rette tid og i rette mængder, men når der er tale om hundreder af forskellige dele og lige så mange mulige varianter af apparaterne kræver det et stort regnskab at sørge for at forsyningerne stadig afpasses efter ordrerne. Da der tillige må regnes med leveringstider for enkeltdelene kræves der et lager, som er kostbart i forrentning. Lagerets størrelse kan imidlertid reduceres gennem et forbedret regnskab. Alle disse omstændigheder gør at det kan blive i høj grad lønnende at lade det centrale lagerog forsyningsregnskab udføre af en datamat. 
}{
Lassen Sie uns nun die Anwendungen der Datenautomaten betrachten, um bei der Überwachung und Steuerung laufender Aktivitäten zu helfen. Zunächst ist hierbei anzumerken, dass eine solche datenbasierte Überwachung keineswegs neu ist, sondern seit vielen Jahren durch die Papier- und Büroarbeit stattfindet, die in den Unternehmen und der Staatsverwaltung durchgeführt wird. Es ist daher nicht überraschend, dass die Datenautomaten zuerst in die laufende Steuerung Einzug erhielten, indem sie einen Teil der Papierarbeit übernahmen, die bisher stattgefunden hatte. Zu den ersten Nutzern von Datenautomaten in diesem Bereich gehörten die Banken. Das Material, mit dem die Banken arbeiten, ist ja größtenteils Geld -- also Daten, die direkt von einem Datenautomaten verarbeitet werden können.

Produktionsunternehmen bieten weitaus variablere Probleme dar, und die Nutzung hochentwickelter \alt{Datenverarbeitung}{Informatik} erfordert eine umfassendere Umstellung. Eines der ersten aufgegriffenen Gebiete war die Lagerbuchhaltung. Diest ist vor allem wichtig bei Produktionen, die darin bestehen, viele Kleinteile zu größeren Geräten zusammenzubauen, zum Beispiel in der Radioindustrie. Gerade weil die Geräte aus vielen Kleinteilen bestehen, ist es für den Hersteller verlockend, viele verschiedene Varianten seiner Produkte anzubieten. Es erscheint ja so einfach, lediglich dafür zu sorgen, dass bestimmte Teile, die in den fertigen Geräten enthalten sind, mit einer passenden Variation ausgewählt werden können -- beispielsweise so, dass ein Radio für mehrere verschiedene Netzspannungen für die Nutzung in verschiedenen Ländern geliefert werden kann. Bei umfangreicheren Produktionen zeigt sich jedoch, dass das Problem, eine Produktion dieser Art zu überwachen, schnell ein erschreckendes Ausmaß annehmen kann. Damit die Produktion gleichmäßig weiterläuft, ist es notwendig, dass die erforderlichen Einzelteile den Monteuren immer rechtzeitig und in der richtigen Menge bereitgestellt werden. Aber wenn es Hunderte verschiedene Teile und ebenso viele mögliche Varianten der Geräte gibt, ist eine komplexe Buchhaltung erforderlich, um sicherzustellen, dass die Lieferungen (der Einzelteile) weiterhin den Bestellungen angepasst werden. Da man zudem mit Lieferzeiten für die Einzelteile rechnen muss, ist ein Lager erforderlich, das in der Finanzierung teuer ist. Die Lagergröße kann jedoch durch eine verbesserte Buchhaltung reduziert werden. All diese Umstände führen dazu, dass es sich in hohem Maße lohnen kann, die zentrale Lager- und Lieferbuchhaltung von einem Datenautomaten durchführen zu lassen.
}

\transSubSec{Produktionsplanlægning}{Produktionsplanung}

\trans{
Lagerregnskabet er dog kun én blandt flere datamatiske opgaver der kan være knyttet til en løbende produktion. En mere indviklet problemstilling træffes ved produktionslinier som man finder dem for eksempel på skibsværfter, hvor de enkelte dele af en produktion passerer fra den ene maskine til den den anden i en bestemt rækkefølge. Skibet består af et stort antal sådanne dele af vidt forskellig størrelse og det tidsrum en del er under behandling ved en bestemt maskine varierer i høj grad. Problemet er at afgøre i hvilken rækkefølge man skal fremstille delene således at de uundgåelige ventetider for maskinerne og arbejderne reduceres mest muligt. Denne principielt så simple opgave er ikke let at løse. Når antallet af maskiner og dele der skal fremstilles ikke er helt lille findes der uhyre mange mulige rækkefølger for produktionen, og man kender ingen simple metoder til at finde frem til den bedste. Når dertil kommer at man til stadighed, for eksempel daglig, har brug for at føre produktionsplanen ajour under hensyn til den faktiske udvikling, så kan det ikke overraske at opgaven kan få datamaterne til at strække ud. 
}{
Die Lagerbuchhaltung ist jedoch nur eine von mehreren datenautomatenbasierten Aufgaben, die mit einer laufenden Produktion verbunden sein können. Eine kompliziertere Problemstellung findet man bei Produktionslinien, wie man sie zum Beispiel auf Werften findet, wo die einzelnen (Einzel-)Teile einer Produktion in einer bestimmten Reihenfolge von einer Maschine zur nächsten gelangen. Das Schiff besteht aus einer großen Anzahl solcher (Einzel-)Teile in sehr unterschiedlicher Größe, und die Zeitspanne, die ein Teil an einer bestimmten Maschine bearbeitet wird, variiert stark. Das Problem besteht darin, die Reihenfolge festzulegen, in der die Teile hergestellt werden sollen, sodass die unvermeidlichen Wartezeiten für die Maschinen und Arbeiter so weit wie möglich reduziert werden. Diese prinzipiell so einfache Aufgabe ist nicht leicht zu lösen. Wenn die Anzahl der Maschinen und der herzustellenden (Einzel-)Teile nicht gering ist, gibt es unglaublich viele mögliche Produktionsreihenfolgen -- und man kennt keine einfachen Methoden, um die beste von ihnen zu finden. Wenn man dann noch berücksichtigt, dass man ständig, zum Beispiel täglich, die Produktionsplanung in Anbetracht der tatsächlichen Entwicklung aktualisieren muss, dann überrascht es nicht, dass diese Aufgabe die Datenautomaten stark beanspruchen kann.
}


\transSubSec{Direkte datamatisk styring}{Direkte datenautomatenbasierte Steuerung}

\trans{
Den opmærksomme lytter har muligvis bemærket den linie der går gennem de tre sidst omtalte problemstillinger, først pengeregnskabet, derefter lagerregnskabet, og sidst den egentlige produktionsplanlægning. Alle tre berører produktionen, men vi er rykket skridtvis ind mod produktionen selv. Dog har vi endnu ved produktionsplanlægningen arbejdere, værkførere, og andre mennesker som mellemled mellem datamatens resultater og den egentlige bearbejdning af råmaterialerne, ligesom vi forudsætter at de data om situationen, som datamaten behøver, stammer fra mennesker. Den konsekvente forlængelse af denne linie er at datamaten selv kobles direkte til passende måleinstrumenter der overvåger produktionen, og til maskinerne, som da styres direkte af datamaten.

En gennemført datamatisk styring kommer i første række på tale ved storproduktion af ensartede produkter. De mest nærliggende eksempler er kemiske produktioner, blandt andet olieraffinering og syntese af kunstgødning. For at yde deres bedste må sådanne store anlæg stadig justeres for at kompensere for de uundgåelige ændringer i ydre vilkår såsom temperatur og råmaterialernes sammensætning. Hvilke justeringer der til enhver tid er nødvendige kan kun bestemmes ud fra et omfattende materiale af data om anlæggets drift, tryk og temperaturer på talrige steder i anlægget. Ved en datamatisk styring lader man alle målinger blive foretaget af elektriske måleinstrumenter der er direkte koblet til datamaten. I datamaten foregår der til stadighed en kontrol af at anlægget holder sig inden for de sikre grænser, at for eksempel temperaturen ikke noget sted stiger over bestemte grænser. Dertil beregnes de justeringer af ventiler og andre styreorganer som er nødvendige for at opnå den mest økonomiske drift. 

Lad os slutte med et eksempel på direkte datamatisk styring der ligger en del nærmere ved de flestes hverdag, og som også ligger på linie med vore eksempler på projektering, nemlig styring af trafik. Ved vejprojektering forsøger man at udforme vejene så godt som muligt, under hensyn til forventede behov. I modsætning hertil affinder man sig ved trafikstyring med vejene som de er, men forsøger at forbedre trafikkens afvikling løbende ved at påvirke trafiklysene og andre signaler til trafikanterne under hensyn til den øjeblikkelige trafiksituation. Et sådant system til datamatisk styring af trafikken har i en årrække været under udvikling i Toronto i Canada. Et stort antal steder i byen er der installeret følere der til enhver tid registrerer trafikstrømmen. Signalerne herfra ledes til en central datamat, som således hele tiden kan have et billede af situationen. Ud fra dette billede styrer datamaten trafiksignalerne således at den forhåndenværende gadekapacitet så godt som muligt stilles til rådighed for det øjeblikkelige trafikbehov. Man behøver ikke at kende meget til omkostningerne ved at udvide gadenettene i eksisterende byer for at forstå at denne måde at udvide kapaciteten kan være endda overordentlig lønnende. 

Lad dette sidste eksempel være en illustration af nødvendigheden af at de der har ansvaret for 
de aktiviteter hvori datamaterne indgår er fuldt fortrolig med datamaternes arbejdsmåde. I anvendelser af denne art er det helt indlysende at datamaten ikke kan opfattes som en tilsats til en iøvrigt lukket konstruktion, men er en helt central bestanddel af systemet. Det vil ikke være muligt at opbygge fuldt datamatisk styrede systemer uden at også dem der udvikler den trafikingeniørmæssige side af sagen har en god fortrolighed med den datamatiske side af sagen. De behøver ikke selv at kunne bygge datamaterne, men de må fuldt ud forstå deres muligheder og begrænsninger og må forstå enkelthederne i den styringsproces datamaten udfører. 

Sagt i korthed vil datamaterne ved enhver anvendelse der går ud over de mest primitive former komme til at sidde ved systemernes livsnerve. Den der vil bevare herredømmet over et system af denne art må beherske datamaten.
}{
Der aufmerksame Zuhörer hat möglicherweise den roten Faden bemerkt, der durch die drei zuletzt besprochenen Problemstellungen verläuft: zuerst die Geldbuchhaltung, dann die Lagerbuchhaltung, und zuletzt die eigentliche Produktionsplanung. Alle drei betreffen die Produktion, aber wir haben uns schrittweise auf die Produktion selbst zubewegt. Doch haben wir bei der Produktionsplanung noch Arbeiter, Vorarbeiter und andere Menschen als Bindeglied zwischen den Ergebnissen des Datenautomaten und der eigentlichen Bearbeitung der Rohmaterialien. Ebenso setzen wir voraus, dass die Daten über die Situation, die der Datenautomat benötigt, von Menschen stammen. Die konsequente Verlängerung dieser Linie besteht darin, den Datenautomaten direkt mit geeigneten Messgeräten zu verbinden, die die Produktion überwachen, und mit den Maschinen, die dann direkt vom Datenautomaten gesteuert werden.

Eine vollständig datenautomatenbasierte Steuerung kommt in erster Linie bei der Massenproduktion einheitlicher Produkte in Frage. Die naheliegendsten Beispiele sind chemische Produktionen, unter anderem Ölraffinierung und die Synthese von Kunstdünger. Um ihre beste Leistung zu erbringen, müssen solche großen Anlagen ständig angepasst werden, um die unvermeidlichen Änderungen der äußeren Bedingungen zu kompensieren, wie Temperatur und die Zusammensetzung der Rohmaterialien. Welche Anpassungen zu einem bestimmten Zeitpunkt notwendig sind, kann nur anhand umfangreicher Daten über den Betrieb der Anlage bestimmt werden, wie Druck und Temperaturen an zahlreichen Stellen in der Anlage. Bei einer datenautomatenbasierten Steuerung lässt man alle Messungen von elektrischen Messgeräten durchführen, die direkt mit dem Datenautomaten verbunden sind. Im Datenautomaten findet fortlaufend eine Kontrolle statt, ob die Anlage innerhalb sicherer Grenzen bleibt, dass also zum Beispiel die Temperatur an keiner Stelle über bestimmte Grenzwerte steigt. Darüber hinaus werden die Anpassungen der Ventile und anderer Steuerorgane berechnet, die notwendig sind, um den wirtschaftlichsten Betrieb zu erreichen.

Lasst uns mit einem Beispiel für direkte datenautomatenbasierte Steuerung abschließen, das dem Alltag der meisten Menschen etwas näher liegt und dennoch auf einer Linie mit unseren Beispielen zur Planung liegt, nämlich der Steuerung des Verkehrs. Bei der Straßenplanung versucht man, die Straßen so gut wie möglich zu gestalten, unter Berücksichtigung des erwarteten Bedarfs. Im Gegensatz hierzu nimmt man bei der Verkehrssteuerung die Straßen so, wie sie sind, aber versucht, den Verkehrsfluss durch Beeinflussung der Ampeln und anderer Signale, unter Berücksichtigung der aktuellen Verkehrssituation, zu verbessern. Ein solches System zur datenautomatenbasierten Verkehrssteuerung wurde seit mehreren Jahren in Toronto, Kanada, entwickelt. An vielen Stellen in der Stadt sind Sensoren installiert, die den Verkehrsfluss jederzeit erfassen. Die Signale werden zu einem zentralen Datenautomaten geleitet, der dadurch ständig ein Bild der Situation haben kann. Auf Basis dieses Bildes steuert der Datenautomat die Verkehrssignale, sodass die vorhandene Straßenkapazität so gut wie möglich dem aktuellen Verkehrsbedarf zur Verfügung gestellt wird. Man muss nicht viel über die Kosten zur Erweiterung der Straßennetze in bestehenden Städten wissen, um zu verstehen, dass diese Methode zur Erweiterung der Kapazität äußerst lohnend sein kann.

Lasst dieses letzte Beispiel eine Veranschaulichung für die Notwendigkeit sein, dass diejenigen, die für die Aktivitäten verantwortlich sind, in die Datenautomaten involviert sind, vollständig mit der Arbeitsweise der Datenautomaten vertraut sein müssen. Bei Anwendungen dieser Art ist es völlig offensichtlich, dass der Datenautomat nicht als Zusatz zu einer ansonsten abgeschlossenen Konstruktion betrachtet werden kann, sondern ein zentraler Bestandteil des Systems ist. Es wird nicht möglich sein, vollständig datenautomatenbasierte Systeme zu entwickeln, ohne dass auch diejenigen, die die verkehrstechnische Seite der Angelegenheit entwickeln, mit der datenautomatenbasierten Seite der Angelegenheit gut vertraut sind. Sie müssen nicht selbst in der Lage sein, die Datenautomaten zu bauen, aber sie müssen deren Möglichkeiten und Grenzen vollständig verstehen und die Einzelheiten des Steuerungsprozesses verstehen, den der Datenautomat ausführt.

Kurz gesagt: Datenautomaten werden bei jeder Anwendung, die über die einfachsten Formen hinausgeht, zur Lebensader der Systeme werden. Wer die Kontrolle über ein solches System behalten will, muss den Datenautomaten beherrschen.
}





