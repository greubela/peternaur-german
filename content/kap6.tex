\transSec{FREMTIDSPERSPEKTIVER}{Zukunftsperspektiven}

\trans{
Vi skal i dag forlænge den linie som allerede har tegnet sig af datamaternes udvikling og brug, og forsøge at få et billede af nogle af de samfundsmæssigt vigtigere muligheder. Jeg skal altså forsøge mig som profet og udsætter mig for risikoen for at blive gjort til grin af eftertiden. Når jeg påtager mig denne tvivlsomme rolle er det fordi vore forestillinger om fremtiden er vigtige for vore dispositioner i nutiden. Jeg vil derfor til slut medtage nogle tanker om hvad vi bør gøre umiddelbart for at datamaterne skal blive til gavn for os alle. 
}{
Heute wollen wir die Linie weiterführen, die sich bereits aus der Entwicklung und Nutzung der Datenautomaten abgezeichnet hat, und versuchen, ein Bild einiger der gesellschaftlich bedeutenderen Möglichkeiten zu zeichnen. Ich werde mich also als Prophet versuchen und setze mich dem Risiko aus, von der Nachwelt ausgelacht zu werden. Wenn ich diese zweifelhafte Rolle auf mich nehme, dann deshalb, weil unsere Vorstellungen von der Zukunft wichtig für unsere Entscheidungen in der Gegenwart sind. Ich werde daher abschließend einige Gedanken einbringen, darüber, was wir unmittelbar tun sollten, damit die Datenautomaten uns allen zugutekommen.
}

\transSubSec{Datamater i privatlivet}{Datenautomaten im Privatleben}

\trans{
For at begynde så anskueligt som muligt vil jeg først overveje udsigterne til at hvermand vil komme i daglig kontakt med datamaterne. Helt konkret kan man spørge om det vil ende med at vi hver især får vor datamat i huset? I betragtning af hvordan datamaterne er reduceret i størrelse er dette ikke udelukket af praktiske grunde. Således har man allerede for længst udviklet datamater der kun fylder som et sædvanligt radioapparat, til brug i fly og raketter. Når vi alligevel ikke skal vente at disse vil vinde almindeligt indpas i dagliglivet er det fordi behovet for isoleret, personlig datamatisk hjælp vil være forholdsvis begrænset. Der er langt mere grund til at vente en udbredt personlig forbindelse til store centrale datamatiske systemer. Situationen ligner den vi kender fra telefon, radio, og fjernsyn. Der er intet i vejen for at etablere rent personlige kommunikationsnet med disse midler, og sådanne findes som bekendt også, men den store udbredelse finder vi hvor de bruges som bindeled mellem den enkelte person og store centrale anlæg. 

Hvis datamatisk tjeneste skal vinde indpas på en analog måde må vi forestille os at den vil virke som en betydelig udvidelse af telefontjenesten. Blandt de etablerede kommunikationstjenester er det jo kun telefonsystemet som byder på den vekselvirkning mellem bruger og system som vil være den væsentlige værdi ved datamatisk tjeneste. Mere konkret kan vi få en forestilling om den sandsynlige udvikling ved at tænke på telefonsystemets særtjenester, især telefonvagten og vækning. 

Hvorvidt en offentlig datamatisk tjeneste vil udvikle sig som en udvidelse af telefonsystemet eller som et uafhængigt system skal jeg lade stå åbent. I alle tilfælde vil brugerne få rådighed over hvad jeg skal kalde en terminal, som muligvis kan være et telefonapparat som vi kender det, men som snarere vil være et apparat som også kan trykke meddelelser på papir og som har taster som en skrivemaskine. Terminaler med en fjernsynsskærm er også en mulighed. Ganske som ved telefonen vil kontakten mellem brugeren og centralsystemet foregå i kortere perioder, betinget af et opkald enten fra brugeren eller fra systemet.
}{
Um so anschaulich wie möglich zu beginnen, werde ich zunächst die Prognose betrachten, dass jedermann täglich mit Datenautomaten in Kontakt kommen wird. Konkret kann man fragen, ob es so weit kommt, dass jeder von uns einen Datenautomaten im Haus haben wird. Angesichts der Tatsache, wie die Datenautomaten in ihrer Größe reduziert wurden, ist dies aus praktischen Gründen nicht ausgeschlossen. So wurden bereits Datenautomaten entwickelt, die nur so viel Platz wie ein gewöhnliches Radio einnehmen, zur Verwendung in Flugzeugen und Raketen. Wenn wir dennoch nicht erwarten sollten, dass diese im täglichen Leben allgemein Einzug erhalten werden, liegt es vermutlich daran, dass der Bedarf an isolierter, persönlicher datenautomatenbasierter Hilfe relativ begrenzt sein wird. Viel eher ist damit zu rechnen, dass es weit verbreitete persönliche Verbindungen zu großen zentralen datenautomatenbasierten Systemen geben wird. Die Situation ähnelt der, die wir vom Telefon, Radio und Fernsehen kennen. Es gibt nichts, was dagegen spräche, rein persönliche Kommunikationsnetze mit diesen Mitteln einzurichten, und solche gibt es bekanntlich auch, jedoch findet die große Verbreitung dort statt, wo sie als Verbindungsglied zwischen dem einzelnen Menschen und großen zentralen Anlagen genutzt werden.

Falls datenautomatenbasierte Dienste auf ähnliche Weise Einzug halten sollen, müssen wir uns vorstellen, dass sie als eine erhebliche Erweiterung des Telefondienstes fungieren werden. Unter den etablierten Kommunikationsdiensten ist es ja nur das Telefonnutz, das die Wechselwirkung zwischen Benutzer und System bietet, die den wesentlichen Wert bei datenautomatenbasierten Diensten darstellen wird. Konkreter können wir uns eine Vorstellung von der wahrscheinlichen Entwicklung machen, indem wir an die Sonderdienste des Telefonsystems denken, insbesondere den Anrufbeantworter und die Weckfunktion.

Ob sich ein öffentlicher datenautomatenbasierter Dienst als Erweiterung des Telefonsystems entwickeln wird oder als ein unabhängiges System, möchte ich offenlassen. In jedem Fall werden die Benutzer Zugriff auf das haben, was ich als Terminal bezeichnen möchte, das möglicherweise ein Telefonapparat sein kann, wie wir ihn kennen, aber eher ein Gerät sein wird, das auch Nachrichten auf Papier drucken kann und Tasten wie eine Schreibmaschine hat. Terminals mit einem Fernsehbildschirm sind ebenfalls eine Möglichkeit. Genau wie beim Telefon wird der Kontakt zwischen dem Benutzer und dem Zentralsystem in kürzeren Zeiträumen stattfinden, abhängig von einem Anruf entweder vom Benutzer oder vom System.
}

\transSubSec{Privat regnskab}{Private Buchhaltung}

\trans{
Hvilken glæde vil en bruger nu kunne få af et sådant system? Det mest nærliggende område er hjælp til at holde styr på brugerens økonomiske forhold. Som en helt primitiv anvendelse kan man nævne beregningen af skatten ud fra oplysninger om indkomst og personlige forhold. Ved denne anvendelse, ligesom ved alle de senere nævnte, må man forestille sig at det datamatiske system vil optræde som en høflig hjælper der stiller spørgsmål om hvad det er brugeren ønsker og som stadig er parat til at forklare nærmere hvis brugeren er i tvivl om hvad det er han skal svare på og som kontrollerer sin forståelse ved bekræftende meddelelser. 

Hjælpen i økonomiske spørgsmål kan udvides til at brugeren kan få råd vedrørende de gunstigste dispositioner, for eksempel de gunstigste lånemuligheder. Så snart vi tænker på sådanne anvendelser bliver det klart hvorfor det er afgørende at tjenesten er knyttet til et fælles offentligt net, og hvorfor helt private datamater ikke frembyder lignende fordele. Sagen er at den fornuftigste disposition er afhængig af en lang række ydre forhold, skattelovgivning, konjunkturer, priser, og så videre. En effektiv, offentlig tjeneste måtte naturligvis stadig holdes ajour om alle disse forhold, hvilket ikke uden videre ville være tilfældet med et uafhængigt system. 

Perspektiverne på dette område rækker dog langt videre. Et centralt datamatisk system vil kunne føre det fuldstændige regnskab for de enkelte brugere. Systemet ville altså kunne overtage de funktioner som allerede nu tilbydes af girokontoret og nogle af bankerne. Betalinger vil derfor kunne foretages uden overføring af penge eller papir, blot gennem en indre dataproces i systemet, som bevirker at betalerens konto for mindskes og modtagerens forøges med beløbet. Vi nærmer os altså et samfund uden repræsentation af pengeværdier i form af mønter og sedler. Herfra er der kun et skridt til en tjeneste som langt mere aktivt hjælper brugeren til at klare de økonomiske forhold. Systemet kunne foretage regelmæssige betalinger efter en besked fra brugeren én gang for alle, kunne udsende opkrævninger og kontrollere at betalingerne gik ind, og kunne holde øje med udviklingen og levere gode råd og forslag om fornuftigere dispositioner. 
}{
Welchen Nutzen könnte ein Benutzer aus einem solchen System ziehen? Das naheliegendste Anwendungsgebiet ist Hilfe bei der Verwaltung der finanziellen Angelegenheiten des Benutzers. Als eine ganz einfache Anwendung kann man die Berechnung der Steuern anhand von Informationen über Einkommen und persönliche Verhältnisse nennen. Bei dieser Anwendung, ebenso wie bei allen später genannten, muss man sich vorstellen, dass das datenautomatenbasierte System als höflicher Helfer auftritt, der Fragen stellt, was der Benutzer möchte, und immer bereit ist, näher zu erklären, wenn der Benutzer Zweifel hat, worauf er antworten soll, und seine Verständigung durch bestätigende Mitteilungen überprüft.

Die Hilfe bei finanziellen Fragen kann erweitert werden, sodass der Benutzer Ratschläge zu den vorteilhaftesten Entscheidungen erhalten kann, zum Beispiel die günstigsten Kreditmöglichkeiten. Sobald wir an solche Anwendungen denken, wird klar, warum es entscheidend ist, dass der Dienst mit einem gemeinsamen öffentlichen Netz verbunden ist, und warum völlig private Datenautomaten nicht ähnliche Vorteile bieten (können). Der Punkt ist, dass die vernünftigste Entscheidung von einer Vielzahl äußerer Faktoren abhängt, wie Steuerrecht, Konjunkturen, Preise und so weiter. Ein effizienter, öffentlich zugreifbarer Dienst müsste natürlich stets über all diese Faktoren auf dem Laufenden gehalten werden, was bei einem unabhängigen System nicht ohne Weiteres der Fall wäre.

Die Perspektiven in diesem Bereich reichen jedoch viel weiter. Ein zentrales datenautomatenbasiertes System könnte die vollständige Buchführung für die einzelnen Benutzer übernehmen. Das System könnte also die Funktionen übernehmen, die bereits jetzt von den Postämtern und einigen Banken angeboten werden. Zahlungen könnten daher ohne Überweisung von Geld oder Papier durchgeführt werden, lediglich durch einen internen Datenprozess im System, wodurch das Konto des Zahlenden verringert und das des Empfängers um den Betrag erhöht wird. Wir nähern uns also einer Gesellschaft ohne Darstellung von Geldwerten in Form von Münzen und Scheinen. Von hier ist es nur ein Schritt zu einem Dienst, der dem Benutzer weit aktiver hilft, seine finanziellen Angelegenheiten zu regeln. Das System könnte regelmäßige Zahlungen nach einer einmaligen Anweisung des Benutzers durchführen, könnte Rechnungen versenden und überprüfen, ob die Zahlungen eingegangen sind, und könnte die Entwicklungen überwachen und gute Ratschläge sowie Vorschläge für sinnvollere Entscheidungen geben.



}

\transSubSec{Privat rådgivning}{Private Beratung}

\trans{
Et system af denne art vil også kunne overtage mange af de simplere informationsformidlinger, som nu ydes af aviser og andre slags skrifter. Hele den kontakt om stillinger og om privat køb og salg af ejendele som nu finder sted gennem dagspressens annoncer kunne overtages af et centralt datamatisk system. Den der vil give et tilbud kunne give mere detaillerede oplysninger end normalt 1 en annonce, og den der ønsker et tilbud kunne ligeledes stille med ret specifikke krav. Systemet kunne da udføre en nøje sammenligning af tilbud og krav og fritage begge parter for at reflektere på muligheder der ligger langt fra ønskerne. 

Det centrale datamatiske system ville også kunne holde øje med om bestemte situationer skulle opstå, om for eksempel en bestemt stilling skulle blive opslået ledig, eller en bestemt film skulle blive sat på programmet i en bestemt biograf, eller et bestemt nummer blive udtrukket i lotteriet. 

Ved en mere radikal udvidelse kan systemet påtage sig en mere almindelig rådgivningsvirksomhed. Som eksempel kan man tænke på sagførernes simplere opgaver. Når vi søger en sagfører er det ofte blot for at sikre os en rådgiver på vor side der har et nøje kendskab til den gældende lov. Der er intet der forhindrer os i selv at læse loven, uden hjælp, men oftest viger vi tilbage derfra fordi det er for besværligt at uddrage de bestemmelser som har betydning for os. Et centralt datamatisk system, som jævnt hen fik tilføjet oplysninger om ændringer i lovgivningen, kunne rådgive borgerne om den gældende lov og dens betydning for den enkelte. 
}{
Ein System dieser Art könnte auch viele der einfacheren Informationsvermittlungen übernehmen, die derzeit von Zeitungen und anderen Arten von Schriften bereitgestellt werden. Der gesamte Kontakt bezüglich Stellenangeboten und dem privaten Kauf und Verkauf von Eigentum, der derzeit durch Anzeigen in der Tagespresse stattfindet, könnte von einem zentralen datenautomatenbasierten System übernommen werden. Derjenige, der ein Angebot machen möchte, könnte detailliertere Informationen geben als normalerweise in einer Anzeige, und derjenige, der ein Angebot wünscht, könnte ebenso recht spezifische Anforderungen stellen. Das System könnte dann einen genauen Vergleich von Angeboten und Anforderungen durchführen und beide Parteien davon entlasten, über Möglichkeiten nachzudenken, die weit von den jeweiligen Wünschen entfernt sind.

Das zentrale datenautomatenbasierte System könnte auch überwachen, ob bestimmte Situationen eintreten würden --- ob zum Beispiel eine bestimmte Stelle ausgeschrieben würde, ein bestimmter Film in einem bestimmten Kino ins Programm aufgenommen würde oder eine bestimmte Nummer in der Lotterie gezogen würde.

Bei einer radikaleren Erweiterung könnte das System eine allgemeinere Beratungstätigkeit übernehmen. Als Beispiel kann man an die einfacheren Aufgaben von Anwälten denken. Wenn wir einen Anwalt aufsuchen, ist es oft nur, um uns einen Berater an unsere Seite zu holen, der ein genaues Wissen über das geltende Recht hat. Es gibt nichts, was uns daran hindert, das Gesetz selbst zu lesen, ohne Hilfe, aber meistens schrecken wir davor zurück, weil es zu mühsam ist, die für uns relevanten Bestimmungen herauszufinden. Ein zentrales datenautomatenbasiertes System, dem regelmäßig Informationen über Gesetzesänderungen hinzugefügt würden, könnte die Bürger über das geltende Recht und dessen Bedeutung für den Einzelnen beraten.
}

\transSubSec{Overvågning af samfundet}{Überwachung der Gesellschaft}

\trans{
De anvendelser jeg her har skitseret kan alle opfattes som hjælp til bedre dispositioner for den enkelte borger i en verden fuld af forandringer. Hvis vi vil forestille os fremtidige anvendelser af datamaterne til væsentlige samfundsmæssige opgaver må vi tænke på den analoge problemstilling, forstørret op til landsomfattende eller international målestok. Her må vi tænke på datamaterne som hjælpemidler ved lovgivningen og den offentlige administration. Med datamaternes hjælp er det i hidtil ukendt målestok muligt at indsamle løbende oplysninger om samfundenes udvikling og at bearbejde disse oplysninger til en form som egner sig som grundlag for den lovgivningsmæssige styring. Denne aktivitet er allerede kendt fra det arbejde som udføres af de offentlige statistiske kontorer, men hidtil har tiden fra indsamlingen af oplysninger til de bearbejdede resultater foreligger løbet op til måneder eller år, og værdien af disse oplysninger for den politiske styring har derfor været begrænset til langtidstendenser, ligesom selve den statistiske bearbejdelse indebærer at de fleste af oplysningernes detailler bortkastes. 

Ved en datamatisk overvågning vil oplysningerne kunne underkastes en langt mere indgående analyse hvorved mere indviklede sammenhænge 1 samfundets dynamik vil kunne opdages. Lad os som eksempel tænke på sundhedsvæsenet. Overblikket over befolkningens sundhedstilstand hviler på oplysninger om den enkelte, som til stadighed indhentes i støre mængder på hospitalerne og hos lægerne. Hidtil har anvendelsen af disse mængder af oplysninger væsentlig været begrænset til at afklare akutte tilstande, mens en systematisk efterbearbejdelse af materialet kun har været foretaget i begrænset omfang. Men paradoksalt nok forholder det sig sådan at efterhånden som man forbedrer den almindelige sundhedstilstand bliver det vanskeligere at efterspore og opklare sygdomme — når hver enkelt læge kun undtagelsesvis får en bestemt sygdomstilstand at se vil det blive vanskeligere at erkende det fælles billede der er grundlaget for at identificere en sygdom og efterspore dens årsag. Med en centraliseret datamatisk opsamling og bearbejdning af oplysningerne om den enkelte borgers sundhedstilstand vil det blive muligt at nyttiggøre dem i langt højere grad, således at man for eksempel tidligere kan efterspore sygdomskilder. 
}{
Die hier skizzierten Anwendungen können alle als Hilfe für bessere Entscheidungen des einzelnen Bürgers in einer Welt voller Veränderungen betrachtet werden. Wenn wir uns zukünftige Anwendungen von Datenautomaten für wesentliche gesellschaftliche Aufgaben vorstellen wollen, müssen wir auch an analoge Problemstellung denken, vergrößert auf landesweite oder internationale Ebene. Hier müssen wir an Datenautomaten als Hilfsmittel für die Gesetzgebung und die öffentliche Verwaltung denken. Mit Hilfe der Datenautomaten ist es in bisher unbekanntem Umfang möglich, laufend Informationen über die gesellschaftliche Entwicklung zu sammeln und diese Informationen so zu verarbeiten, dass sie als Grundlage für die gesetzgeberische Steuerung geeignet sind. Diese Tätigkeit ist bereits von der Arbeit der öffentlichen Statistikämter bekannt, aber bisher hat die Zeitspanne von der Datensammlung bis zu den vorliegenden aufbereiteten Ergebnissen Monate oder Jahre betragen, und der Wert dieser Informationen für die politische Steuerung war daher auf langfristige Trends beschränkt, ebenso wie die statistische Bearbeitung selbst bedeutet, dass die meisten Details der Informationen verworfen werden.

Bei einer datenautomatenbasierten Überwachung könnten die Informationen einer weitaus gründlicheren Analyse unterzogen werden, wodurch komplexere Zusammenhänge in der Dynamik der Gesellschaft erkannt werden könnten. Lasst uns als Beispiel an das Gesundheitswesen denken. Der Überblick über den Gesundheitszustand der Bevölkerung beruht auf Informationen über den Einzelnen, die fortlaufend in großen Mengen in Krankenhäusern und bei Ärzten erhoben werden. Bisher war die Nutzung dieser Informationsmengen im Wesentlichen auf die Klärung akuter Zustände beschränkt, während eine systematische Nachbearbeitung des Materials nur in begrenztem Umfang durchgeführt wurde. Paradoxerweise verhält es sich jedoch so, dass es schwieriger wird, Krankheiten nachzuverfolgen und aufzuklären, je mehr sich der allgemeine Gesundheitszustand verbessert, wenn jeder einzelne Arzt nur ausnahmsweise einen bestimmten Krankheitszustand zu Gesicht bekommt. Dies erschwert es, das gemeinsame Bild zu erkennen, das die Grundlage für die Identifizierung einer Krankheit und die Nachverfolgung ihrer Ursache bildet. Mit einer zentralisierten datenautomatenbasierten Erfassung und Verarbeitung der Informationen über den Gesundheitszustand des einzelnen Bürgers wäre es möglich, sie in weit höherem Maße zu nutzen, sodass man zum Beispiel früher Krankheitsursachen nachverfolgen kann.
}

\transSubSec{Sikring af privatlivets fred}{Schutz der Privatsphäre}

\trans{
Den overvågning af borgerne som bliver mulig i det datamatiserede samfund indebærer den fare at oplysningerne vil blive misbrugt, altså at oplysninger der med befolkningens viden og ønske er blevet indsamlet til visse formål, imod dette ønske bliver benyttet af mindre grupper som våben mod andre grupper eller individer. Farerne herved er allerede kommet frem ved de misbrug af efterretningsvæsenets kartoteker som har været omtalt i dagspressen i de senere år, og det fremgår heraf at faren ikke udelukkende er knyttet til datamatiske kartoteker. Ved brug af datamater vil det imidlertid blive muligt at holde styr på flere oplysninger om flere borgere, og det vil blive lettere at søge frem til bestemte oplysninger. Der ligger heri en fristelse for administrationen til at ophobe oplysninger, for det tilfælde at de senere skulle vise sig at være nyttige. Omfanget af et muligt misbrug kan herved øges uhyggeligt. 

Af denne grund må det være klart at etable- - ringen af offentlige datamatiske medborgerregistre bør ledsages af en omhyggelig lovgivning til at forhindre misbrug. Blandt de forholdsregler denne lovgivning bør omfatte kan især nævnes tre, For det første bør tilgangen af oplysninger til registret i videst mulige omfang begrænses til oplysninger der er godkendt af den pågældende borger, eller som borgeren har haft mulighed for at imødegå ved hjælp af en offentligt beskikket forsvarer. Enhver borger må have ret til at betragte enhver oplysning om hans person der indgår i registret som et vidneudsagn der kan blive brugt imod ham og må sikres et sædvanligt retsligt forsvar. For det andet bør systemet være opbygget omkring en nøje klassifikation af oplysningerne, svarende til hemmeligholdelsesgraden, med en tilsvarende graduering af den kontrol der finder sted ved henvendelser udefra med anmodning om at oplysningerne udleveres. Helt almindeligt må det gælde at fremskaffelsen af dybt hemmelige oplysninger må kræve en mere besværlig og tidsrøvende kontrol af at henvendelsen er berttiget. For det tredie bør systemet være forberedt på i en ekstraordinær situation med kort varsel at destruere visse dele af registret, efter ordre fra regeringen. 

På den anden side må det ikke overses at datamaterne på grund af deres systematiske arbejdsmåde yder den fordel at det er muligt at sørge for at én gang fastlagte administrative kontrolprocedurer virkelig stadig overholdes, uden at man er meget afhængig af fortsat menneskelig påpasselighed. Det springende punkt er den oprindelige udformning af systemet. Her er der brug for en kombination af indsigt i fremkommelige fremgangsmåder til hemmeligholdelse som man må kunne hente hos den militære sagkundskab, dertil indsigt i retsbeskyttelsen, som den kan ydes fra juridisk hold, og endelig viden om datamatiske muligheder. En gruppe med denne indsigt måtte kunne udarbejde forslag som basis for et politisk valg. Betingelsen for at vi skal bevare herredømmet er i denne sag, som i alle tilsvarende, at de der har ansvaret for udformningen virkelig forstår de processer datamaterne udfører for os.
}{
Die Überwachung der Bürger, die in der datenautomatenbasierten Gesellschaft möglich wird, birgt die Gefahr, dass die Informationen missbraucht werden, also dass Informationen, die mit dem Wissen und dem Wunsch der Bevölkerung zu bestimmten Zwecken gesammelt wurden, gegen diesen Wunsch von kleineren Gruppen als Waffe gegen andere Gruppen oder Einzelpersonen verwendet werden. Die Gefahren sind bereits durch den Missbrauch der Akten des Nachrichtendienstes ans Licht gekommen, der in den letzten Jahren in der Tagespresse erwähnt wurde $^{(6.1)}$ und daraus ergibt sich, dass die Gefahr nicht ausschließlich mit datenautomatenbasierten Karteien verbunden ist. Durch den Einsatz von Datenautomaten wird es jedoch möglich, mehr Informationen über mehr Bürger zu verwalten, und es wird einfacher, bestimmte Informationen zu finden. Darin liegt eine Versuchung für die Verwaltung, Informationen zu horten, für den Fall, dass sie sich später als nützlich erweisen könnten. Das Ausmaß eines möglichen Missbrauchs kann dadurch erschreckend vergrößert werden.

Aus diesem Grund muss klar sein, dass die Einrichtung öffentlicher datenautomatenbasierter Bürgerregister von einer sorgfältigen Gesetzgebung begleitet werden sollte, um Missbrauch zu verhindern. Unter den Maßnahmen, die diese Gesetzgebung umfassen sollte, können insbesondere drei genannt werden. Erstens sollte der Zugang zu den Informationen im Register so weit wie möglich auf Informationen beschränkt werden, die von dem betreffenden Bürger genehmigt wurden oder denen der Bürger mithilfe eines öffentlich bestellten Vorsprechers entgegentreten konnte. Jeder Bürger muss das Recht haben, jede Information über seine Person, die im Register enthalten ist, als Zeugenaussage zu betrachten, die gegen ihn verwendet werden könnte, und muss eine übliche rechtliche Verteidigung gewährleistet bekommen. Zweitens sollte das System um eine genaue Klassifizierung der Informationen herum aufgebaut sein, entsprechend dem Geheimhaltungsgrad, mit einer entsprechenden Abstufung der Kontrolle bei Anfragen von außen, die die Herausgabe der Informationen verlangen. Ganz allgemein sollte gelten, dass die Beschaffung von streng geheimen Informationen eine aufwendigere und zeitintensivere Überprüfung der Berechtigung der Anfrage erfordern muss. Drittens sollte das System darauf vorbereitet sein, in einer außergewöhnlichen Situation mit kurzer Vorankündigung bestimmte Teile des Registers auf Anordnung der Regierung zu vernichten.

Andererseits darf nicht übersehen werden, dass Datenautomaten aufgrund ihrer systematischen Arbeitsweise den Vorteil bieten, dass es möglich ist sicherzustellen, dass einmal festgelegte administrative Kontrollverfahren wirklich weiterhin eingehalten werden, ohne dass man stark von anhaltender menschlicher Sorgfalt abhängig ist. Der entscheidende Punkt ist die ursprüngliche Gestaltung des Systems. Hier ist eine Kombination aus Einsicht in praktikable Methoden zur Geheimhaltung erforderlich, die man von militärischem Fachwissen ableiten kann, dazu Einsicht in den Rechtsschutz, wie er von juristischer Seite geboten werden kann, und schließlich Wissen über datenautomatenbasierte Möglichkeiten. Eine Gruppe mit dieser Einsicht müsste in der Lage sein, Vorschläge als Grundlage für eine politische Entscheidung zu erarbeiten. Die Voraussetzung dafür, dass wir die Kontrolle behalten, ist in diesem Fall -- wie in allen entsprechenden Fällen -- dass diejenigen, die für die Gestaltung verantwortlich sind, die Prozesse, die die Datenautomaten für uns ausführen, wirklich verstehen.
}

\transSubSec{Datamaternes fremtrængen}{Das Vordringen der Datenautomaten}

\trans{
Som det sidste emne for disse forelæsninger vil jeg nu overveje de problemer der på dette område umiddelbart rejser sig for vore erhversvirksomheder og for vort land som helhed. Det er ikke nogen tilfældighed at disse problemer kan slås sammen under ét. Med en vis rimelighed kan man betragte et land som en virksomhed i større skala, som står overfor andre lande på samme måde som virksomhederne står overfor hinanden, i et forhold præget såvel af indbyrdes afhængighed som af konkurrence. 

Ved generelle overvejelser på dette niveau må vi først og fremmest interessere os for styring, altså kombinationen af en løbende indsamling af data om verdens gang og en kombinering af disse data til en beslutning om hvad vi i hvert øjeblik skal gøre for at forløbet skal følge en ønske lig retning. Det første vi må indse er at vi, det vil sige både vore virksomheder og den offentlige forvaltning, før eller siden vil blive tvunget over i at bruge datamatisk styring. Dette følger direkte af konkurrencen. Omkostningerne ved at gennemføre en vis dataproces med en datamat vil nemlig efter alt at dømme endnu i lang tid fortsætte ad den hastigt faldende kurve, der i løbet af de sidste fem år har reduceret prisen til en tiendedel. Samtidig er omkostningerne ved at gennemføre de samme dataprocesser med menneskelig arbejdskraft stigende. 

Den næste væsentlige omstændighed ved datamatikkens fremtrængen er at omstillingen til datamatisk styring nødvendigvis må foregå under ledelse af de pågældende virksomheder selv, hvad enten disse er private eller offentlige. Det er en farlig misforståelse at tro at denne nye teknik kan købes som en færdig pakke, som man for eksempel i en virksomhed køber en lastbil. Styringen af en virksomhed berører hele beslutnings- og magtfordelingen inden for virksomheden. Udformningen af styringssystemet kan ikke overlades til udenforstående uden at evnen til efter eget ønske at ændre på styringssystemet, og i sidste instans selvstændigheden, går tabt. På det nationale plan berøres selve lovgivningen. 

Virksomhederne og den offentlige administration vil altså blive tvunget til at indføre datamatisk styring og må selv lede udviklingen hver på deres område. Men hertil behøver de yderligere støtte fra specialister på området datalogi og datamatik. Som den tredie væsentlige omstændighed ved situationen må fremhæves at denne støtte, for at være effektiv, må kunne hentes hos lokale kyndige. Indarbejdelsen af datamatisk styring i en virksomhed kræver en lang overgangsperiode præget af et nært samarbejde med datamatiske eksperter, og dette samarbejde vil hæmmes af alle geografiske og sproglige forskelle. En af forudsætningerne for at denne støtte kan være til rådighed er at datalogien vinder indpas i uddannelserne, både alment og som specialstudium. 

Placeringen af et fag i uddannelsen alene behøver ikke at være tilstrækkelig til at sikre det den støtte der er ønskelig fra et samfundssynspunkt. Det mest fremtrædende udslag af denne erkendelse inden for vore grænser er etableringen af atomenergikommissionens forsøgsanlæg ved Risø. Begrundelsen for Risø er at de øvrige energikilders utilstrækkelighed inden for en overskuelig tid vil tvinge os til at gå over til atomenergi. For at bevare en rimelig national uaf hængighed på dette område er det nødvendigt at vi råder over første rangs sagkundskab der stadig kan vurdere mulighederne ud fra vort synspunkt. Men en sådan sagkundskab kan kun erhverves og holdes vedlige af folk der stadig kan arbejde med forskning og udvikling på området. Denne virksomhed finder sted på Risø. 
}{
Als letztes Thema dieses Vorlesungsteils möchte ich nun die Probleme betrachten, die sich in diesem Bereich unmittelbar für unsere Unternehmen und für unser Land als Ganzes ergeben. Es ist kein Zufall, dass diese Probleme unter einem Punkt zusammengefasst werden können. Man kann mit einer gewissen Berechtigung ein Land als ein Unternehmen im größeren Maßstab betrachten, das anderen Ländern in gleicher Weise gegenübersteht wie Unternehmen einander, in einem Verhältnis, das sowohl durch gegenseitige Abhängigkeit als auch durch Konkurrenz geprägt ist.

Bei allgemeinen Überlegungen auf diesem Niveau müssen wir uns in erster Linie für Steuerung interessieren, also die Kombination einer laufenden Sammlung von Daten über den Lauf der Welt und einer Kombination dieser Daten zu einer Entscheidung darüber, was wir zu jedem Zeitpunkt tun sollen, damit der Verlauf einer gewünschten Richtung folgt. Das Erste, was wir erkennen müssen, ist, dass wir -- das heißt sowohl unsere Unternehmen als auch die öffentliche Verwaltung -- früher oder später gezwungen sein werden, datenautomatenbasierte Steuerung einzusetzen. Dies folgt direkt aus dem Wettbewerb. Die Kosten für die Durchführung eines bestimmten Datenprozesses mit einem Datenautomaten werden aller Voraussicht nach noch lange dem steil abfallenden Kurvenverlauf folgen, der in den letzten fünf Jahren den Preis (bereits) auf ein Zehntel gesenkt hat. Gleichzeitig steigen die Kosten für die Durchführung derselben Datenprozesse mit menschlicher Arbeitskraft.

Die nächste wesentliche Gegebenheit beim Vordringen der \alt{Datenauotmatik}{Informatik} ist, dass die Umstellung auf datenautomatenbasierte Steuerung notwendigerweise unter der Leitung der betreffenden Unternehmen selbst erfolgen muss, ob es sich um private oder öffentliche Unternehmen handelt. Es ist ein gefährliches Missverständnis zu glauben, dass diese neue Technik als fertiges Paket gekauft werden kann -- wie man zum Beispiel in einem Unternehmen einen Lastwagen kauft. Die Steuerung eines Unternehmens betrifft die gesamte Entscheidungs- und Machtverteilung innerhalb des Unternehmens. Die Gestaltung des Steuerungssystems kann nicht an Außenstehende überlassen werden -- sonst verliert man die Fähigkeit, dieses Steuerungssystem nach eigenem Wunsch zu ändern, und letztlich die Unabhängigkeit. Auf nationaler Ebene wird sogar die Gesetzgebung selbst berührt.

Die Unternehmen und die öffentliche Verwaltung werden also gezwungen sein, datenautomatenbasierte Steuerung einzuführen und müssen die Entwicklung jeweils in ihrem Bereich selbst leiten. Dazu benötigen sie jedoch zusätzliche Unterstützung von Spezialisten auf dem Gebiet der \alt{Datalogie und Datenverarbeitung}{Informatik}. Als dritte wesentliche Gegebenheit der Situation muss hervorgehoben werden, dass diese Unterstützung, um effektiv zu sein, von lokalen Fachleuten bezogen werden muss. Die Einführung der datenautomatenbasierten Steuerung in ein Unternehmen erfordert eine lange Übergangszeit, geprägt von einer engen Zusammenarbeit mit datenautomatenkundigen Experten, und diese Zusammenarbeit wird durch alle geografischen und sprachlichen Unterschiede behindert werden. Eine der Voraussetzungen dafür, dass diese Unterstützung zur Verfügung stehen kann, ist, dass die \alt{Datalogie}{Informatik} in die Ausbildung Einzug erhält -- sowohl allgemein als auch als Spezialstudium.

Die Aufnahme eines Fachs in die Ausbildung allein muss nicht ausreichen, um die Unterstützung sicherzustellen, die aus gesellschaftlicher Sicht wünschenswert ist. Der auffälligste Ausdruck dieser Erkenntnis innerhalb unserer (dänischen) Grenzen ist die Einrichtung der Versuchsanlage der Atomenergiekonmission in Risø. Die Begründung für Risø ist, dass uns die Unzulänglichkeit der übrigen Energiequellen in absehbarer Zeit zwingen wird, auf Atomenergie umzusteigen. Um eine angemessene nationale Unabhängigkeit auf diesem Gebiet zu bewahren, ist es notwendig, dass wir über erstklassiges Fachwissen verfügen, welches weiterhin die Möglichkeiten von unserem Standpunkt aus bewerten kann. Aber ein solches Fachwissen kann nur von Personen erworben und aufrechterhalten werden, die weiterhin in Forschung und Entwicklung auf diesem Gebiet tätig sind. Diese Tätigkeit findet in Risø statt.
}

\transSubSec{Nationerne ruster sig}{Die Nationen rüsten sich} 

\trans{
I samfundsmæssig betydning står datalogien på ingen måde tilbage for studiet af atomenergien. I U.S.A. hvor der er en stærk gensidig påvirkning mellem samfundet og teknikkens og forskningens resultater, har dette for længst givet sig udslag i en vældig aktivitet omkring datamaterne. De europæiske lande, først og fremmest England, stod omkring 1950 fuldt på højde med U.S.ÅA., men er siden gledet bagud, i den forstand at man i stadig stigende omfang har importeret amerikanske datamater og har været afhængig af amerikanske ideer, og tager man området som helhed ligger U.S.A. klart forud for den øvrige verden. Blandt årsagerne hertil kan man utvivlsomt nævne den større kapitalkoncentration i U.S.A., men dette dækker slet ikke hele sagen. Langt vigtigere er formentlig den fleksibilitet der præger tankegangen og in stitutionerne, deriblandt universiteterne, i U.S.A. Den traditionalisme og stivhed der præger de europæiske universiteter, med deres faste stabe af urørlige professorer, har på dette, som på mange andre områder, voldt de europæiske lande umådelig skade. 

I adskillige europæiske lande findes der dog nu en erkendelse af at denne udvikling betyder en alvorlig trussel mod vore samfund og man træffer tegn på en vilje til at dæmme op for den voksende amerikanske dominering på området. I Frankrig har man således set at amerikanske opkøb af franske datamatfabrikker er blevet besvaret med en statsstøtte til opbygning af en konkurrencedygtig fransk industri. Denne støtte beløber sig til over 600 millioner kroner over årene 1967 til 71. I England har man besluttet sig til at forbedre universiteternes udrustning med datamater for et beløb af 600 millioner kroner over de næste seks år. I Norge har staten besluttet sig til at bruge omkring 50 millioner kroner i løbet af fem år. 
}{
In gesellschaftlicher Bedeutung steht die \alt{Datalogie}{Informatik} dem Studium der Atomenergie in keiner Weise nach. In den USA, wo es eine starke gegenseitige Beeinflussung zwischen der Gesellschaft und den Ergebnissen der Technik und Forschung gibt, hat sich dies längst in einer enormen Aktivität rund um die Datenautomaten niedergeschlagen. Die europäischen Länder, allen voran England, waren um 1950 voll auf Augenhöhe mit den USA, sind aber seither zurückgefallen -- in dem Sinne, dass man in zunehmend größerem Umfang amerikanische Datenautomaten importierte und von amerikanischen Ideen abhängig war. Betrachtet man das Gebiet insgesamt, liegen die USA eindeutig vor dem Rest der Welt. Unter den Gründen hierfür kann man zweifellos die größere Kapitalkonzentration in den USA nennen, aber das erklärt keineswegs die ganze Sache. Weitaus wichtiger ist vermutlich die Flexibilität, die das Denken und die Institutionen -- darunter auch die Universitäten -- in den USA kennzeichnet. Der Traditionalismus und die Starrheit, die die europäischen Universitäten mit ihren festen Stäben von unantastbaren Professoren prägen, haben auf diesem, wie auf vielen anderen Gebieten, den europäischen Ländern enormen Schaden zugefügt.

In mehreren europäischen Ländern gibt es jedoch inzwischen die Erkenntnis, dass diese Entwicklung eine ernsthafte Bedrohung für unsere Gesellschaften darstellt, und es gibt Anzeichen für den Willen, der zunehmenden amerikanischen Dominanz auf diesem Gebiet entgegenzuwirken. In Frankreich hat man beispielsweise gesehen, dass amerikanische Aufkäufe französischer Datenautomatenfabriken mit staatlicher Unterstützung für den Aufbau einer wettbewerbsfähigen französischen Industrie beantwortet wurden. Diese Unterstützung beläuft sich auf über 600 Millionen Kronen in den Jahren 1967 bis 1971 $^{(6.2)}$. In England hat man beschlossen, die Ausstattung der Universitäten mit Datenautomaten in den nächsten sechs Jahren um einen Betrag von 600 Millionen Kronen zu verbessern. In Norwegen hat der Staat beschlossen, in den nächsten fünf Jahren etwa 50 Millionen Kronen zu investieren $^{(6.3)}$.
}

\transSubSec{Danmarks situation}{Die Situation Dänemarks}

\trans{
I sammenligning hermed er det der hidtil er gjort af den danske stat meget beskedent. Det er så meget mere trist som både vor hjemlige produk tion af datamater og den brug der er gjort af disse datamater i både private og statslige virksomheder har været fuldt ud konkurrencedygtige. En stor del af problemet må søges i en traditionel mangel på kontakt og deraf følgende mistillid mellem på den ene side statsvirksomhederne, specielt universiteterne og centraladministrationen, og på den anden side industrivirksomhederne. Når disse blandes med nationale mindreværdskomplekser og en tilsvarende blåøjet tillid til amerikansk teknik kan man komme i den besynderlige situation at vore højere læreanstalter afviser konstruktive tilbud om samarbejde fra dansk industri, ud fra en frygt for usaglige motiver bag tilbudet, hvorefter man kaster sig i armene på et amerikansk gigantforetagende, hvis kommercielle motiver ingen har grund til at tvivle om. 

Hvad vi har brug for er to ting. For det første en kraftig styrkelse af uddannelsen i datalogi på alle niveauer, ikke mindst ved de højere læreanstalter. Udviklingen hemmes her ganske væsentligt af uddannelsessystemets traditionelle stivhed. Der er dog en voksende erkendelse af at en usædvanlig indsats er påkrævet, omend resultaterne stadig lader vente på sig. Det er også værd at nævne at de uafhængige erhvervsvirk somheder har følt mangelen på uddannelsesmuligheder så stærkt at de for nylig har dannet et erhvervenes uddannelsesfond for elektronisk databehandling, med betydelig økonomisk rygdækning. 

Den anden ting vi har brug for er at fremhjælpe vor uafhængighed på området, ved at yde statsstøtte til hjemlig forsknings- og udviklingsvirksomhed, således at vi vil få en gunstig jordbund for en uafhængig udnyttelse af mulighederne. For at en sådan støtte skal blive gavnlig for landet som helhed er det vigtigt at det arbejde der gøres stadig holdes i kontakt med de dele af erhvervslivet der kan drage nytte deraf. Af denne grund er tanken om en uafhængig statslig forskningsinstitution, et datalogiens Risø, mindre frugtbar. Hvad man snarere måtte overveje er en stærk udvidelse af mulighederne for at yde støtte til sådanne erhvervsvirksomheder, som selv er interesserede i at arbejde med at udvikle nye datamatiske løsninger inden for deres eget felt, til at disse virksomheder kunne erhverve sig den nødvendige indsigt gennem efteruddannelse af medarbejdere og gennem samarbejde med uafhængige datalogiske specialistvirksomheder. I betragtning af hvad der hidtil er nået herhjemme kunne en sådan støtte forventes at give endda overordentlig gode resultater. Det bør også nævnes at den form for støtte der her anbefales allerede nu praktiseres gennem det erhversforskningsfond der er etableret af Danmarks teknisk-videnskabelige Forskningsråd. 

Lad os resumere: vi skal ikke vente at datamaterne vil påvirke den ydre form for vort dagligliv så meget som for eksempel bilerne eller fjernsynet. Virkningen ligger dybere, i vor tankegang og den måde vi går i lag med problemerne i virksomhederne og i det offentlige maskineri. Specielt vil forandringerne gælde administration og ledelse, som står over for dybtgående ændringer i arbejdsform. Hvad specielt angår Danmarks muligheder for at hævde sig, så har vi svagheder, men også stærke punkter, og situationen giver os grund til behersket optimisme.
}{
Im Vergleich dazu ist das, was der dänische Staat bisher unternommen hat, sehr bescheiden. Das ist umso trauriger, weil sowohl unsere heimische Produktion von Datenautomaten als auch deren Einsatz in privaten und staatlichen Unternehmen voll konkurrenzfähig gewesen sind. Ein großer Teil des Problems liegt in dem traditionellen Kontaktmangel und daraus resultierendem Misstrauen zwischen den staatlichen Unternehmen auf der einen Seite -- insbesondere den Universitäten und der Zentralverwaltung -- und den Industrieunternehmen auf der anderen Seite. Wenn dies mit nationalen Minderwertigkeitskomplexen und einem ebenso blauäugigen Vertrauen in die amerikanische Technik vermischt wird, kann man in die merkwürdige Situation geraten, dass unsere Hochschulen konstruktive Kooperationsangebote der dänischen Industrie aus Angst vor unsachlichen Motiven hinter dem Angebot ablehnen, woraufhin man sich in die Arme eines amerikanischen Großunternehmens stürzt, dessen kommerzielle Motive niemand anzweifeln muss.

Was wir brauchen, sind zwei Dinge. Erstens eine deutliche Stärkung der Ausbildung in der \alt{Datalogie}{Informatik} auf allen Ebenen, nicht zuletzt an den Hochschulen. Die Entwicklung wird hier erheblich durch die traditionelle Starrheit des Bildungssystems gehemmt. Es gibt jedoch ein wachsendes Bewusstsein dafür, dass ein außergewöhnlicher Einsatz erforderlich ist, obwohl die Ergebnisse noch auf sich warten lassen. Es ist auch erwähnenswert, dass die unabhängigen Unternehmen den Mangel an Ausbildungsmöglichkeiten so stark empfunden haben, dass sie kürzlich einen Ausbildungsfonds der Wirtschaft für elektronische Datenverarbeitung gegründet haben - mit erheblicher finanziellen Mitteln.

Das zweite, was wir brauchen, ist die Förderung unserer Unabhängigkeit auf diesem Gebiet durch staatliche Unterstützung für heimische Forschungs- und Entwicklungsarbeit, sodass wir einen günstigen Boden für eine unabhängige Nutzung der Möglichkeiten schaffen. Damit eine solche Unterstützung für das Land als Ganzes nützlich ist, ist es wichtig, dass die geleistete Arbeit weiterhin in Kontakt mit den Teilen der Wirtschaft steht, die davon profitieren können. Aus diesem Grund ist die Idee einer unabhängigen staatlichen Forschungseinrichtung, einem Risø der \alt{Datalogie}{Informatik}, weniger fruchtbar. Vielmehr sollte man eine deutliche Erweiterung der Möglichkeiten zur Unterstützung solcher Unternehmen in Betracht ziehen, die selbst daran interessiert sind, neue datenautomatenbasierte Lösungen in ihrem eigenen Bereich zu entwickeln, damit diese Unternehmen das notwendige Wissen durch Weiterbildung ihrer Mitarbeiter und durch Zusammenarbeit mit unabhängigen \alt{datalogischen}{informatischen} Spezialunternehmen erwerben können. Angesichts dessen, was bisher im Land erreicht wurde, könnte eine solche Unterstützung voraussichtlich sogar außerordentlich gute Resultate erzielen. Es sollte auch erwähnt werden, dass die hier empfohlene Art der Unterstützung bereits durch den Forschungsfonds für die Wirtschaft praktiziert wird, der vom (besagten) Dänischen Technisch-Wissenschaftlichen Forschungsrat eingerichtet wurde.

Lasst uns zusammenfassen: Wir sollten nicht erwarten, dass Datenautomaten die äußere Form unseres täglichen Lebens so sehr beeinflussen werden wie zum Beispiel Autos oder Fernseher. Die Wirkung liegt tiefer, in unserem Denken und der Art, wie wir die Probleme in den Unternehmen und in der öffentlichen Verwaltung angehen. Besonders werden die Veränderungen die Verwaltung und Führung betreffen, die vor tiefgreifenden Änderungen in der Arbeitsweise stehen. Was speziell Dänemarks Möglichkeiten betrifft, sich zu behaupten, so haben wir Schwächen, aber auch starke Punkte, und die Situation gibt uns Grund zu verhaltenem Optimismus.
}